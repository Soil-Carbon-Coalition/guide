\documentclass[11pt,letterpaper,twoside,onecolumn]{memoir}
\usepackage{utopia,graphicx,textcomp,makeidx,lettrine,color,wrapfig,hyperref}
\usepackage[small]{caption}
\usepackage{comment}
\usepackage{enumitem,amssymb,ulem}
\newlist{checkboxlist}{itemize}{1}
\setlist[checkboxlist]{label=$\square$}

\usepackage{tcolorbox}

\widowpenalty = 8000
\clubpenalty = 2000
\hyphenpenalty=1000
\linespread{1.20}
\sloppy


\makeindex

%LAYOUT
\settypeblocksize{9.2in}{6in}{*}
\setulmargins{.8in}{*}{*}
\setlrmargins{*}{*}{*}
\setheaderspaces{*}{.27in}{*}
\checkandfixthelayout


\definecolor{shadecolor}{gray}{0.64}

%this is for small sidebars, do not put across pagebreaks
%\newenvironment{WrapText}[1][r]
%  {\wrapfigure{#1}{0.5\textwidth}\tcolorbox}
%  {\endtcolorbox\endwrapfigure}

%for sharp cornered boxes
%\begin{tcolorbox}[width=\textwidth,colback={red},title={With true corners},outer arc=0mm,colupper=white]    
%text goes here
%\includegraphics[scale=0.5]{frogimage.png}
%\end{tcolorbox} 



\renewcommand{\captionfont}{\footnotesize}
\begin{document}

\frontmatter
\pagestyle{empty}

%TITLE


\begin{center}
\Huge
\bfseries
A field guide\\
to the most powerful and\\
creative planetary force




\begin{figure}[h]
\includegraphics[width=\textwidth]{pics/elephant.pdf}
\end{figure}

\vspace*{1em}

\Large
\bfseries
flexible, practical, local\\
paths to engagement

\vspace*{2em}
Peter Donovan




\footnotesize{version: April 2017}
\begin{figure}[h]
\centering
\includegraphics[width = 1in]{pics/creativecommons1.png}
\end{figure}

This guide can be freely copied and adapted,\\with attribution, no commercial use, and\\derivative works similarly licensed.



\end{center}

\newpage
\thispagestyle{empty}
\setlength{\parindent}{1em}
\setlength{\parskip}{0em}
\normalsize
\setlength{\epigraphwidth}{4in}
\setlength{\epigraphrule}{0pt}
\epigraphfontsize{\small}
\setlength{\beforechapskip}{0em}


\tableofcontents*
\clearpage

\chapter{What this guide is about, \\and how to use it}

\epigraph{If you want to make small changes, change how you do things. But if you want to make big changes, you need to change how you see things.}{Don Campbell}

\noindent The major challenges and opportunities we face, as individuals, communities, and as civilizations, have to do with how we see things. What do we recognize or fail to recognize, and how do we act as a result? 

When I was in my twenties, I had one of my first sheep herding jobs on a backcountry ranch in Idaho. I had charge of a thousand ewes with their lambs, living out of a tent camp. It was late spring. The boss told me to move my sheep from the river canyon where they had been grazing, up a long steep grassy hillside to the higher country where we would spend the summer.

What I had learned was that to move sheep, you got behind them and pressured or drove them in the desired direction. My little border collie dog was a big help, her quick to-and-fro helping me drive them in the direction I was facing. In the morning I pushed the sheep off the bed ground onto the bottom of the hill. The day grew hot. They would graze up the hill a ways, and then drift back down to shade up under the pine trees along the river. At midday, progress on the uphill move was zero. Late in the afternoon I tried again, pushing and driving. As the day began to cool, the sheep were finally stringing out up the long hill. If I kept pressuring the laggards at the bottom, it looked like I might get the whole band up to the top, where my tent and bedroll and food now was, by nightfall.

Far up the hill, the leading sheep started bunching up and flowing back down toward me. With binoculars I could see a large black-and-white border collie expertly turning the sheep. They flowed down the hill, lines of sheep merging into a cream-colored descending flow. I had seen this dog at the ranch, and they had told me about old Tweed: deaf and indifferent to commands, and sometimes would travel miles to gather sheep.

But I saw my day's effort unraveling. I yelled, I even fired my rifle into the rocks above him, but the sheep kept flowing downhill. Failure. I would have to spend the night with the sheep on the river, without bed, tent, or food.

Years later I went to Bud Williams's superb workshop on low-stress livestock handling. Bud emphasized the crucial difference between making animals do things, and letting them do them, between what we want to do and what we need to do to be successful, and how much better results you get from the latter. Bud showed and explained how you could position yourself to \textit{let} animals do what you needed them to do, rather than trying to \textit{make} them or force them to do what you wanted them to do.

During a break I recounted the Tweed incident to Bud, fishing for some kind of expert answer. Bud said, ``You were out of position.''

\textit{Out of position.} What did that mean? I was expecting to be told what I was doing wrong, how to do it right, perhaps some strategies and actions I might have taken to manipulate or solve the Tweed problem. But out of position?

I started remembering, and even seeing things differently. Shifts of perspective and belief can be transitory, easily lost or reversed. But given the circumstances and cues, they can also be renewed or regained. Later that summer, I had had occasional trouble keeping the sheep from scattering too far in those rugged mountains. On two occasions Tweed came out and gathered them for me. I was thrilled with his power, attitude, and range of position, and how easily the sheep flowed and collected in response. He saved me many hours of hard work. In big rough country, the best kind of herding dog is one that will bring livestock to you. Tweed was that kind, and a great one.

In that first encounter, I had not understood Tweed's power, and so had mistaken an opportunity for a threat. This kept me in problem-solving mode, out of position, in a downward spiral of reaction, of shrinking possibilities and choices.

Allan Savory and others later introduced me to holistic decision making, and I began to write and report on people and groups who were achieving integrated social, ecological, and economic successes by managing wholes, managing \textit{for} what they needed, and positioning themselves accordingly. I began to realize that the carbon cycle---the circle of life---was the most powerful and creative planetary force.

But this power and creativity could remain camouflaged, out of focus, or appear as some kind of threat. Our predominant human orientation is around threats and problems, managing \textit{against} what we don't want. Our institutions reflect this. Expert knowledge, policy, markets, and incentive systems were out of position, ill-suited to recognize possibilities for outgrowing those problems by working with the circle of life, farming or ranching in nature's image.

With this in mind I started the Soil Carbon Challenge in 2010, a monitoring project focused on possibility, on feedback, on learning how to work with the circle of life. If we want to know how fast a human can run 100 meters, I reasoned, we don't build a computer model, search the relevant literature, or convene experts to make a prediction. We don't engage in an endless series of abstract speculations or definitions. We run a series of races, seeking possibility in specific, local, measured events.

Since then I've met more and more people who grasp the need for a grounded, shared, and shareable intelligence on soil health and watershed function, and who are realizing that such intelligence is not coming from theories, models, predictions, branded advocacies, Best Management Practices, or institutional research, well-intentioned as many of these efforts are. This guide is for them.

It is about the fundamental processes that make our world go round. Not the actual physical spin, but how sunlight, especially when captured by green plants, drives everything that is important to us and our civilizations. It is about gauging fundamental biosphere function at specific locations. It is more about learning and discovery of possibility, than about defending or validating existing knowledge or attacking conventional wisdom. It is about getting up close and personal with the most powerful and creative planetary force, monitoring changes over time.

This guide attempts to cut through some of the confusion and technical trappings that have accumulated around the subject of ecosystem or landscape function, soil carbon change, and measurement. It outlines monitoring methods that are flexible, adaptable, practical, and have the potential to change the way you see things.

The first of three parts describes the work and power of the biosphere, energized by sunlight. Our planet, while a closed system for matter, is an open system for energy, out of equilibrium, and full of possibilities and potentials. Soils are the center of gravity, and this brings opportunities for change in science and leadership.

The second part describes some simple field methods for assessing and measuring this work and power, at local and management scales.

The third part outlines some activities, many of them suitable for classroom use, that help people get hands-on experiences of the basic processes of carbon and water cycling. These activities are an excellent fit with the National Science Education Standards in the U.S., which emphasize systems thinking.

%Notes on sources, and further references, begin on pages \pageref{notes} and \pageref{references} respectively.
 
This guide is neither definitive nor authoritative, but an ongoing process and effort. The ways that living organisms do work, and the ways of observing and measuring that work in accessible and practical ways, have been and are being discovered by many creative and insightful people. We invite you to contribute suggestions, measurement methods, activities, and lesson plans that may help to engage people with local and specific changes in landscape function. And please check back periodically for the latest version at \url{http://soilcarboncoalition.org/guide}

In developing this guide (an ongoing process) I am indebted to hundreds of dedicated and hardworking people, who have both taught me some possibilities about measuring soil carbon change and other aspects of biological work, and have helped me understand the questions, methods, possibilities, contexts, and limitations both of the opportunity of enhancing landscape function, and in our ways of thinking about it. Thanks to Doug McDaniel of Lostine, Oregon for creating opportunity and time for me to write along the banks of the river, and for emphasizing all along that if you want to achieve results on the land, find successful managers and learn from them. And thanks to Didi Pershouse for so much wise guidance in improving the accessibility and presentation of this guide.

\vspace{1em}
\hfill Peter Donovan


\mainmatter

%\makepagestyle{myheadings}
\pagestyle{myheadings}
\renewcommand{\chaptermark}[1]{\markboth{#1}{}}
\makeheadrule{myheadings}{\textwidth}{\normalrulethickness}
\makeevenhead{myheadings}{\thepage}{\footnotesize{A field guide to the most powerful planetary force}}{\reflectbox{\includegraphics[height=15pt]{pics/elephantoutline1.pdf}}}
\makeoddhead{myheadings}{\includegraphics[height=15pt]{pics/elephantoutline1.pdf}}{\footnotesize{Creation is now}}{\thepage}


\makechapterstyle{mybook}{
\renewcommand\chaptitlefont{\normalfont\Huge\bfseries\raggedright}
\setlength{\beforechapskip}{3em}
\renewcommand{\printchaptername}{}
\renewcommand\chapternamenum{}

\renewcommand\printchapternum{%
\makebox[\textwidth][r]{\hspace{0pt}%
\resizebox{!}{6ex}{\chapnamefont\bfseries\thechapter}}}
\renewcommand\afterchapternum{\par\hspace{1.5cm}\hrule\vskip\midchapskip}
}

%TITLE page

\addcontentsline{toc}{part}{Part 1: The circle of life}
\Huge{Part 1: The circle of life}
\normalsize
\thispagestyle{empty}
\vspace*{5 em}
\begin{figure}[h]
\includegraphics[width=\textwidth]{pics/pedicab.JPG}
\end{figure}



\chapterstyle{mybook}

\chapter{Seven generations of sunlight}


\epigraph{L'Homme est la nature prenant conscience d'elle-m\^{e}me.\\\textit{Humankind is nature becoming self-conscious.}}{Elis\'{e}e Reclus, \textit{L'Homme et la Terre} (1905)}


\noindent Have you ever played in a flowing stream? Have you placed rocks, sticks, or sand to alter the flow, or built a dam or levee?

You get to try things, and stuff happens---maybe not what you expect. Creation and destruction are constant. There is turbulence. Sometimes a little bit of work, such as moving a rock, can result in a big change, as the stream scours the bed and deposits sediment somewhere else. Other times, lots of work may result in little change, as the stream, with time, reverts back to the original channel.

By comparison, playing in a quiet puddle or a pond is routine and predictable. Ripples and waves disappear unless you work to keep them going. To make big changes, you need to do lots of work, move a lot of material. 

We live, work, and play in the midst of flows of matter and energy. And we \textit{are} flows of matter and energy, like a flame, a river, a whirlpool, eddies or curlicues in these larger flows. 

Though we lose a little hydrogen to space, and gain some meteorites, the total mass of our earth is basically stable. More locally, tons of water, soil, and other stuff flow through our parcels of land and our watersheds. Driving these movements of matter are huge flows of energy.

Our whole planet is an open system for energy. This is an inconceivably large flow. Earth is receiving gazillions of watts or horsepower from the sun, and pouring it back into space as ``waste heat.''

\begin{figure}
\includegraphics[width=\textwidth]{pics/degreesKelvin.jpg}
\caption*{The surface of the sun is about 5800 degrees Kelvin (degrees Celsius above absolute zero) although its interior temperatures are in the millions. Most of the solar radiation we receive at the surface of the earth is visible light. Earth averages about 13 degrees Celsius above the freezing point of water. Sooner or later we re-radiate nearly all the energy we receive from the sun as longwave or infrared radiation into cold, dark space.}
\end{figure}

So what is this flow or transfer of energy? How does sunlight make things happen? How do we, flickering flames who are part of it, play with it?

Energy is often described as the capacity to do work. In other words, we know it or sense it by its results. Work is force over distance, or force against resistance. Work can produce change, or in some cases can slow, stop, or even reverse change---for example when a refrigerator is turned on to cool food that was just heated.

Power is the rate of work, or work per unit of time. In the 1770s James Watt found that a brewery horse, by turning a shaft, could lift 180 pounds 180 feet in one minute. This became known as a horsepower, and it is about 746 watts (32,400 foot-pounds per minute).

There is a flow of heat from the center of the earth, which is still hot from the coalescence, and to which the fission of radioactive elements contributes. But at the surface where we live, this flow of energy is extremely minor in most places, about the same intensity as moonlight. There is also some work done by tides, the interactions of the gravitational fields of moon and sun as the earth revolves.

But almost all power comes to us as electromagnetic radiation from our nearby star, whose nuclear furnace puts out a very big number of watts. The wattage that reaches earth is also a big number. To get a human-scale perspective, we'll use watts per square meter of the earth's surface, and horsepower per acre. 

These numbers are approximate of course, averaged over day and night, all seasons, all latitudes.\label{watts per square meter} As the average human has approximately one ovary and one testicle, it is worth keeping in mind that averages may be rare, or brief and transitory such as average temperature or rainfall, and that solar energy at a particular time in a particular place may be above or below the average value, often significantly. But these approximate numbers help to give a perspective on the flow of energy, which is often camouflaged from us by our orientation around problems, scarcities, and threats. 

\section*{1. Reflection and atmospheric absorption}

Sunlight energy is not mechanical energy, but arrives as electromagnetic radiation, mostly light of various wavelengths. An average of about 340 watts per square meter of the earth's surface, or about 1,850 horsepower per acre, reaches the top of our atmosphere. About a third of this radiation does no work, but is merely reflected back into space by clouds, snow and ice, aerosols, dust particles, and other reflective surfaces. Some is absorbed in the atmosphere by various gases that are mostly transparent to visible light but absorb some wavelengths, and re-radiate it in all directions as heat (commonly known as greenhouse gases). These include water vapor, carbon dioxide, methane, nitrous oxide, and a few others. Reflection, atmospheric absorption, and re-radiation from these atmospheric gases result in an average of about 240 watts per square meter reaching the surfaces of the earth: ocean, rock, soil, a blade of grass, your cheek. If all this could be converted to mechanical power, it would be about 1,300 horsepower per acre. The brewery horses would be shoulder to shoulder.


\section*{2. Surface absorption}

About two-thirds of what reaches the earth's surface, averaging about 160 watts per square meter or 870 horsepower per acre, is absorbed as heat, much of it by dark oceans. Because the earth is round, varied, and spins on a tilted axis relative to the sun, heat absorption is uneven. This drives gazillions of horsepower worth of ocean currents and winds, and moves equatorial heat toward higher latitudes.

\begin{figure}[h]
\centering
\begin{minipage}{.80\textwidth}
\includegraphics[width=\textwidth]{pics/oceancurrents.jpg}
\caption*{The patterns of ocean currents give us a marvelous picture of the response of our planet to sunlight, with the unpredictable, turbulent, and shifting interplay of necessity and chance. For NASA's beautiful timelapse video, see \url{http://managingwholes.com/ocean-currents}}
\end{minipage}
\end{figure}


\section*{3. Water cycling}

The other third of what reaches the surface, averaging about 80 watts per square meter or 435 horsepower per acre, evaporates water from seas, soils, and plant tissues. This drives water cycling.

As water moves from the solid or liquid phase to a vapor, it absorbs and stores an immense amount energy in the form of heat. It becomes lighter than air, moves and rises, and releases this heat on condensation or freezing. If it falls on land, it moves back to the sea in rivers, groundwater, or ice. Water cycling spreads heat from the equator toward the poles. It is an enormous flow of matter and energy.

Water, in vapor form, is also the principal heat-absorbing gas in the earth's atmosphere. Without it, as John Tyndall realized in 1859, the warmth absorbed during the day ``would pour itself unrequited into space'' at night, and we would not have a climate suitable for life.\label{water} Water cycling moderates and adds another dimension to our climate, and adds considerable variability in weather.

The water cycle functions as a kind of heat engine with sunshine supplying the power by evaporating water, and spontaneously falling liquid water providing the motive force. Over the ocean, falling water doesn't leave an obvious history---although salinity differences from evaporation and rainfall over the oceans can contribute to currents.

Over land it is a different story. Falling and moving water and ice has enormous mechanical power, and sculpts the landscape, sometimes in bizarre and unique ways. Once a drop of water has followed a path, it is easier for other drops to follow, creating rills and channels into which its power is concentrated, with cumulative and accelerating results in erosion and sedimentation. The movement of water over millions of years has carved Grand Canyons and built the terraces and plains on which our cropland soils have developed.

Water also carries dissolved salts and other substances, including dissolved carbon dioxide (carbonic acid) from the air which slowly dissolves rocks. Combined with freezing and thawing, which cracks rocks, water cycling is a powerful agent of geologic change, contributing to the movement and circulation of enormous amounts of matter.

In combination with heat absorption, winds, and currents, the water cycle produces weather and climate. It also has produced massive transformation of land surfaces through weathering, erosion, and deposition by water and ice.

\begin{figure}
\includegraphics[width=\textwidth]{pics/7generations.pdf}
\caption*{\textbf{Seven generations of sunlight.} Sunlight energy is represented by descending lines, some of which is reflected or scattered. Complex feedback loops and possible patterns or paths of influence are represented by the irregular cloud. Because the whole system is \textbf{not} in static equilibrium, there is sensitivity and possibility for change.}
\end{figure}

\section*{4. Carbon cycling: photosynthesis}

Plants eat light. Though photosynthesis uses only a pinhole's worth of the sunlight that reaches earth's surface, it does creative chemistry that over time transforms the flows of matter and energy on earth. This doesn't happen spontaneously. Creative chemistry requires high-quality energy, not just warmth, to sever chemical bonds and assemble new ones. Reactants must be concentrated and often catalyzed, which requires a container such as the membranes of the chloroplasts and cell walls where the reactions take place.

Carbon cycling involves two complementary chemical reactions: photosynthesis and oxidation. In photosynthesis, the chlorophyll pigment absorbs solar photons and uses their energy to split water molecules. The oxygen is given off into the atmosphere, and the hydrogen is combined with the carbon and oxygen from carbon dioxide to build carbohydrates, the stuff and fuel of life.

In oxygen respiration, which is the reverse reaction that completes the carbon cycle, the carbohydrates are oxidized (that is, they combine with oxygen) and the products are carbon dioxide, water, and energy. When this spontaneous reaction takes place outside a cell, we call it oxidation or fire.

This solar-powered chemical work cycle, mediated by carbon atom's chemical bonding with either oxygen or hydrogen or both, powers almost all life and its actions, including your eye moving across this page. (Exceptions to solar power, but not to carbon cycling, are the sulfate-reducing bacteria who inhabit hot undersea vents.) It connects living organisms in a mutually dependent web of energy flow. 

%(\label{discovery}For a review of some of the major discoveries of carbon cycling, see the note on page \pageref{discoverynote}.)

For living organisms, this carbon cycle or circle of life couples growth and decay, love and death, production and consumption, engine and fuel, order and freedom, actuality and possibility. For the planet as a whole, it creates and maintains a radical chemical disequilibrium or potential, like a charged battery, exemplified by our atmosphere with abundant oxygen and low carbon dioxide, that we have traditionally viewed as the nonliving environment of life. The creative power of this chemical potential is immense.

Current photosynthesis has been estimated to be about $1.3 \times 10^{14}$ watts (130 terawatts), 174 billion horsepower, averaging slightly more than 1 horsepower per acre of the earth's surface, about .25 watts per square meter. This is about eight times the world power consumption by humans, which today is mainly dependent on ancient photosynthesis in the form of fossil fuels. Geologic processes such as volcanoes, sedimentation, and rock weathering move carbon to and from the atmosphere but in an average year this is only a small fraction of the carbon turnover of the biosphere.

\begin{figure}
\includegraphics[width=\textwidth]{pics/carboncycle.jpg}
\caption*{The major flows of carbon in the biosphere. Fossil fuel burning (far left) represents only a few percent of the annual flux of carbon dioxide to the atmosphere. The geological carbon cycle is likewise just a small bit of the huge cycle of carbon driven by photosynthesis and biology. (See Rattan Lal, Sequestration of atmospheric CO$_{2}$ in global carbon pools, \textit{Energy and Environmental Science} 1, 86--100 [2008].)}
\end{figure}

The subsequent ``generations'' of sunlight outlined here are eddies or whorls of the carbon cycle, driven by the work of photosynthesis. 

\section*{5. Carbon cycling: respiration, behavior, knowing}

Most of the chemical energy produced by photosynthesis, and stored in chemical bonds of organic carbon compounds, is respired (oxidized) in the mitochondria, the energy transducers, of living cells. A fair amount is burned in immense and numerous fires that occur on every continent except Antarctica. Some is stored or buried in some longer-lasting forms such as soil carbon, fossil fuel deposits, or in marine sediments.

Behavior is a general term for what self-motivated living organisms \textit{do} with this energy, which is turned into proteins, heat, activity, motion. Another general term is cognition or knowhow.\label{knowhow} Much of this is biochemical, such as enzymes working on their target molecules, bacteria moving along a chemical gradient, plants initiating flowering in response to changing day length, or when you become adrenalized by sudden fear or transformed by love. Part of this behavior is the building of the photosynthetic molecules and structures themselves, so immediately we see that there are profound circularities in the organization of living beings---they literally make themselves, using energy and materials from their surrounds. A cell membrane is able to distinguish between different elements or ions, and behave accordingly, and regulate passage of ions. This is knowing or cognition, but it is not necessarily conscious in our usual sense, or even necessarily associated with a nervous system or brain.

Every action, every expenditure of energy, every development by living organisms is a manifestation of knowing. Knowing = doing and vice versa. As an embryo, you \textit{knew} how to develop five fingers on each hand. This wasn't theoretical knowledge. You probably weren't aware of the exact steps you took to do this. And yet you can pass this knowing, or knowhow, to your kids.

Knowing or cognition is a powerful coupling of an organism to its surrounds, but it is not a mechanical system, or a one-way interaction between subject and object. As Gregory Bateson noted, it's a very different thing to kick a stone, and to kick a dog. The stone responds according to the mechanical energy from your foot, and the dog responds with energy from its metabolism, influenced also by patterning from experience.

The behavior or cognition of living organisms also leads to diversification, because every living being is interacting with its surrounds, which include changing populations of other living beings. Behaviors vary from many causes, and these variations may be conserved or further diversified in future generations, or the organism may become extinct, as have most of the world's multicellular life forms.

Amongst all life, humans have become a powerful geologic and planetary force. Human metabolism, which fuels our actions, thoughts, and feelings, runs at about .7 terawatts (700 gigawatts, or $7 \times  10^{11}$ watts) worldwide, slightly over half a percent of the estimated power generated by photosynthesis. Globally this raw human power is multiplied many times by our use of fossil fuels and other energy sources to power our technology (this is also our behavior, totaling about 17 terawatts).

\section*{6. Consciousness, language, beliefs}

Knowing \textit{that} we know is a faculty of humans and quite likely some other mammals and even birds.  This is a sense or perception of a self, being a witness to one's mental processes and states. One strong indicator is language, which enables us (and perhaps commits us) to talk to ourselves as well as to each other. We reflect on our experience and formulate our beliefs, what we think we know, using language, telling stories. Once humans formed intelligent nests, using language, we were able to modify our environment with purpose and anticipation. 

Language and the stories we tell ourselves and each other play a key role.

Our perceptions, beliefs, or knowledge of the circle of life, is also part of that circle. We are not ``outside'' observers, but participants.


\section*{7. Self-awareness}

is consciousness about consciousness---knowing \textit{how} we know, \textit{how} we make decisions, the extent and character of our ignorance, and some awareness of our beliefs, paradigms, and assumptions. This awareness may not be easy or routine. Consciousness, as E. O. Wilson remarks, was not designed for self-examination. Our beliefs and assumptions are often not even visible to us until they fail. When we do ``see'' how it is we see, when we recognize our hidden beliefs, it opens the way for creation, innovation, transcendence, and self-transformation.

As many have noted, there can be no firm distinction between the knower and the known, between subject and object, but all are embedded in relatedness. This error of regarding subject and object as distinct is difficult to escape. The error is intangible and slyly embedded in our language, our beliefs, and most of our science.  Wrote Gregory Bateson, ``it is as if you had touched honey. As with honey, the falsification gets around; and each thing you try to wipe it off on gets sticky, and your hands remain sticky.''

Like other life processes, self-awareness may be a very slight use of solar-derived energy, but it generates enormous cumulative effects. Changes in how we see ourselves and our power relative to the biosphere as a whole, how we make decisions, recognize or fail to recognize threats and opportunities, and how we select and organize leadership are, and will continue to be, key influences on the power and uses of sunlight on earth.

\vspace*{1em}
\begin{center}

\S
\vspace*{1em}
\end{center}

\noindent These cascades of sunlight are a flow of energy with many eddies and whorls. Energy or work is not a thing, but a process. It occurs over time.

The chemical energy of living matter, including the food you eat, is eventually turned into heat by decay processes, respiration, or fire. This heat, along with the other 240 watts per square meter reaching the surface, is re-radiated to space as longwave or infrared radiation.

The earth is currently absorbing slightly more than is re-radiated because of increasing heat-absorbing, heat-trapping gases in our atmosphere such as water vapor and carbon dioxide. James Hansen\label{half watt} estimated this difference at about .58 watt per square meter. In other words, about a quarter of a percent of the power that reaches the earth's surface is currently retained, mainly in the surface layers of the ocean, raising the average surface temperature of the earth.


\section*{Creation is now}

\epigraph{Force is in the long run dissipative; chance is in the long run concentrative. The dissipation of energy by the regular laws of nature is by those very laws accompanied by circumstances more and more favorable to its reconcentration by chance.}{C.S. Peirce (1839--1914)}

\noindent The larger uses of sunlight, such as surface heat absorption and water cycling, are huge. But because they are moving and flowing, not in some kind of static equilibrium, they can be sensitive.

Say you want to make waves in a still, unruffled lake. It takes effort or power to generate these waves. Once you stop, time will soon erase your work, and the lake will be still again.

With a system in motion, not in a static\index{equilibrium} equilibrium, things can be very different. In a fast stream, you can place a boulder or log that will create waves, eddies, and pools, scour the bed here and deposit sediment there, and even change the stream's course. Instead of erasing your efforts, time may amplify them. You share control, and co-create change, with the flow.

The flow of sunlight energy has eddies and turbulence, which then moderate and diversify it, and act as feedbacks. In the dissipation or spreading out of sunlight energy, there are concentrations, and you are one of them: your metabolism, in watts per kilogram, is about 10,000 times the watts per kilogram that the sun generates. 

Encyclopedias could be written about these influences and feedbacks. The following is a short list, grossly incomplete:

\begin{itemize}

\item[] Winds and currents modify clouds and ice, and planetary reflectance.

\item[] Water cycling spreads heat from the equator toward the poles, influencing winds and currents as well as clouds and ice. Water cycling produces atmospheric water vapor which is the main heat-trapping gas, and affects planetary reflectance with clouds.

\item[] The photosynthetic branch of the carbon cycle modifies and slows water cycling, particularly on land. Plants, with their partners below and above ground, have created water-holding fertile soils. Surface litter, and biologically produced sticky macromolecules that hold soil aggregates together, slow down the huge, potentially destructive force of moving water and moderate stream flow. Plant transpiration also modifies local water cycles, and the nucleation of raindrops is influenced by many biologic processes.\label{biotic pump}

\item[] The behavior of microbes, plants, animals, including the diversity and abundance thereof, conditions carbon cycling and the composition of the atmosphere. Carbon dioxide in the atmosphere traps heat, and speeds up water cycling.

\item[] Human consciousness and beliefs, particularly widely shared ones, have an enormous influence on the behavior of plants, animals, and microbes, in large part through agriculture, including how we manage or influence grazing animals.

\item[] Human self-awareness---knowing how we know, being conscious of our beliefs and having the capacity to adapt them to our real needs and gifts---uses a tiny trickle of the flow of sunlight but could have the greatest leverage of all.
\end{itemize}

\noindent All of this becomes circular. The larger powers dominate, but the smaller powers have influence out of proportion to their energy. For example, human decisions now influence even the first generations of sunlight---reflectance, heat absorption, and water cycling. Though some of these effects are unintended and/or undesired, they point to the possibility of intentionally working with these huge powers, of symbiosis rather than the degradation that has so often been the consequence of the human pursuit of sustenance.

In the early stages of earth, the residual heat of formation and the heat generated from meteoric impacts was likely substantial and transformative. But today these forces are slight. The flux of heat from the earth's interior (some from radioactive decay, some residual heat from the earth's formation) averages only about .06 watts per square meter. Radioactive decay, with refined fuels, also powers nuclear reactors. Sunlight reflected to earth from the moon contains little energy, but tidal forces can be large.

\begin{wrapfigure}[28]{R}{6cm}
\centering
\vspace{-2em}
\begin{framed}
\includegraphics[width=5cm]{pics/powergraph.pdf}
\caption*{The relative power of three planetary forces, in terawatts (trillion watts, or watts $\times 10^{12}$). From left to right, geothermal heat flux (44), photosynthesis (130), and worldwide nonfood energy consumption (17). The great power of photosynthesis is amplified even further, on land, by its buffering effect on water cycling.}
\end{framed}
\end{wrapfigure}



\begin{figure}
\begin{tcolorbox}
\setlength{\parskip}{.7em}

Forty years ago \textbf{E. F. Schumacher} distinguished two behaviors or processes that occur when we engage with problems. One behavior begins with many ideas and converges to a best solution, or a small number of best solutions. The example he gave was the problem of an efficient, human-powered wheeled vehicle, and the solution almost universally converged on was the bicycle. This phenomenon or behavior was typical in technical matters, where problems could be clearly defined and had clear boundaries, and we got clear and fairly rapid feedback on what worked and what didn't.

The other behavior was divergence. Here people respond to a situation with different and contradictory views and practices. No single best idea or solution emerges and sweeps the field. This is typical with complex issues, such as relationships and interactions among self-motivated living organisms and their changing environments, habitats, or communities. The boundaries or definitions of a problem are vague, variously defined or framed. There might be time lags between action and results, and the results might support multiple interpretations of cause and effect.

With divergence, a problem doesn't get fixed or solved, especially at a global level. But it might be transcended, outgrown, or shifted to a bigger, wider, more inclusive context---and at first this might happen locally. Conflict can be a creative force, especially if it is treated or facilitated as such, opening some new, larger grasp of the situation and its opportunities. To some extent this is happening today with the issue of human influence on the rest of nature. We question more and more the necessity and advisability of human degradation of the biosphere in the pursuit of sustenance, while also recognizing the insufficiency of an environmental agenda that has mainly sought to wreck the planet slower. An interesting mix of third alternatives have arisen and are continuing to arise in many places, typically from the margins of power and influence.

What seems to stall growth or transcendence and perpetuate gridlock is when we expect or demand convergent behavior even in situations of complexity---when we expect a winner, best practice, or game changer to emerge, like the vaccine for polio. This expectation is wide and deep. We want an answer, something that ties cause to effect in ways that correspond to our beliefs, our identities, our customary strategies and actions, and our organizational habits of program delivery.

Divergent situations are also known as \textit{wicked problems} in some sectors, and are the natural domain of holistic decision making (Savory and Butterfield 1998), as well as consensus building (Chadwick 2012).

Eisenhower said when a problem can't be solved, enlarge it. Yet we typically do the opposite, analyzing a problem into smaller and smaller pieces, because this is how we know to take action. It works well for troubleshooting an engine. But it does not work with complex, interconnected living communities---though we have ingrained habits, institutions, and reward systems that keep us trying.

How do we enlarge our problems, and enlist this greater power of the biosphere? How do we learn to grow and tend this power in a creative and perhaps even collaborative way, to manage \textit{for} what we want and need? Can we develop answers to these questions that go beyond empty words or models, that can be implemented, and are already being implemented, by actual people?

\end{tcolorbox}
\end{figure}

Planetwide, we are in near equilibrium as far as amount of matter. But we're an open system for energy flow.

Yet even as we recognize the complexity and interconnectedness of major issues and challenges, the illusion persists: of stability or static equilibrium, of a world of objects that we judge good or bad, a world of linear cause and effect in which knowing and consciousness are some kind of facsimile of an external reality, a world where power is not shared, of controllers versus the controlled. Many of us believe that the world was fixed long ago, that the past is more powerful than the present, and that things are hard to change, with massive inertia that we cannot overcome or influence, unless it is to throw things out of balance. Even our own beliefs and behaviors appear to have enormous inertia.

Each generation of sunlight adds linkages and feedback loops to the work of the others, resulting in ungraspable complexity. While our knowledge may be increasing, our ignorance of how this all works may be growing faster. Because we are riding a river of sunlight, we are in process of becoming, and creation is now.  \label{chance} We live in a landscape of possibility, with continual fluctuations, but there is no guarantee of progress in any certain direction. Time is not merely a cost or an urgency, but also an opportunity for the cumulative effects of what we are doing to show and unfold.

For the most part, we are using, restraining, or redirecting our 17 terawatts of technological power to manage \textit{against} what we don't want. We are trying to manage parts rather than wholes, things rather than relationships, using what E. F. Schumacher called convergent problem-solving (see sidebar). When applied to complex situations, this kind of problem-solving breeds other, bigger problems and unintended consequences, as well as conflict, power struggles, hierarchy, and turf-guarding. The resulting gridlock leaves most of our major problems apparently insoluble. Many feel powerless, without hope.

Given that the most powerful, transformative, and creative planetary force is sunlight capture by self-motivated living organisms, what might the implications be for our politics and institutions? for leadership? for science and monitoring? for the politics of environmental action?

\chapter{Soils are the center}

\epigraph{Humus plays a leading part in the storage of energy of solar origin on the surface of the earth.}{Selman Waksman, \textit{Humus} (1936)}

\noindent Many of us view soils as a dance floor or stage---the more stable and inert, the better---for the main show featuring above-ground plants, animals, buildings, and machines. But these are appearances. 

Huge flows of matter and energy are moving through soils. Much of the traffic is transparent gases. Water is most of it, but carbon, nitrogen, and other elements are flowing through, into, out of, and across soil. Human knowledge of these patterns and dynamics is a long way from complete, and many significant discoveries await, but here are some generalizations:

\textbf{Carbon cycling.} Soils contain more carbon than plants and atmosphere combined, mostly in the form of carbon-rich organic compounds that have been called soil organic matter, humus, or organo-mineral complexes. These compounds and complexes are incessantly being created through photosynthesis, which splits water and forms carbohydrates, using CO$_{2}$ from the atmosphere. The carbohydrates and their derivatives are incessantly being oxidized by many forms of digestion, decay, and combustion, sooner or later returning CO$_{2}$ to the atmosphere. 

One of the main ways plants put carbon into soil is by leaking liquid sugars and other carbohydrates from their roots. These are used as food by mycorrhizae, bacteria, and in turn by many other small soil-dwelling organisms that form the soil foodweb. In turn these communities of organisms make mineral nutrients and water available to plant roots. (These soil-forming communities, including mycorrhizae, helped plants to colonize land surfaces 400 million years ago.) The residue or litter that plants shed onto the soil surface as their leaves fall, or as they go dormant or die, helps protect the soil from exposure to sun, sudden extremes of temperature, and pounding raindrops. Most of this litter or residue sooner or later will end up as CO$_{2}$ in the atmosphere, but while it's on the surface it offers protection, food, and habitat to multitudes of decomposer organisms. The soil carbon or soil organic matter in soils consists of the living, the dead, and the very dead. It is over half carbon by dry weight. 

\begin{figure}
\begin{framed}
\centering
\includegraphics[width=\textwidth]{pics/soilthecenter.pdf}
\caption*{\begin{minipage}{0.45\textwidth}
Major flows in the fast, active \textbf{carbon cycle} on land.
\end{minipage}
\hfill
\begin{minipage}{0.45\textwidth}
Major flows of \textbf{water cycling}. Soil moisture is the hub for terrestrial life, including the economies and infrastructures of human civilizations.
\end{minipage}}
\caption*{In these simplified diagrams, fuzzy boundaries are made sharp and minor flows are not included. The sizes of the pools or reservoirs are approximate, roughly proportional to the order of magnitude of estimated size.}
\end{framed}
\end{figure}

On land, soils are also the hub of \textbf{water cycling.} Soils contain lots of water in the form of soil moisture---more water than atmosphere, living plants, and rivers combined. This water moves, at different rates. Where the soil surface is exposed to the sun and can heat nearly to the boiling point, an invisible river of water vapor can move skyward from an area of exposed soil, and an entire soil profile can be dried out in a short time. Where soil is protected by a layer of residue or mulch, there is much more effective capture and retention of water from precipitation. Soil cover and soil structure are often ignored or bypassed in our mental models of water cycling, and this is what infiltration demonstrations and rainfall simulators (in parts 2 and 3 of this guide) attempt to remedy. (See \url{http://managingwholes.com/eco-water-cycle.htm} for animations and videos.)

Soils are the bridge that enables carbon cycling---the circle of life---to influence and buffer the enormously greater power and energy of the water cycle. Without carbon cycling and the growth of living tissue, there wouldn't be anything to slow down water. Rains would wash soil into the sea in a blink of geologic time. Worldwide, soil compaction---the loss of soil structure, which accelerates both runoff and evaporation---has contributed significantly to sea level rise and climate change. 

\begin{figure}
\begin{tcolorbox}
\setlength{\parskip}{.7em}
\textbf{Carbon cycling: atmosphere, oceans, and rocks}

The oceans contain much more carbon than soils and atmosphere combined, mostly in the form of carbonate ions (CO$^-$). About half of global photosynthesis takes place in the oceans, mostly by single-celled organisms which get their carbon from these carbonate ions in the water. Most of this carbon is oxidized or respired through the oceanic food chain. The oceans freely exchange carbon dioxide with the air, and act as a kind of buffer on atmospheric carbon dioxide, absorbing CO$_{2}$ (and becoming slightly more acid) when there is higher concentration in the air, and releasing it when the air has less, much like how a carbonated beverage adapts its carbonation to the concentration or partial pressure of CO$_2$ in the air above it. Because of this ocean buffer, changing the atmospheric concentration of CO$_{2}$ has been, and will continue to be, a fairly long and drawn-out process in human terms.

The global carbon endowment includes even larger amounts of carbon in rocks, but most of it is relatively slow-moving and low-powered in comparison to life's quick turnovers using reduction-oxidation reactions. Sedimentation over eons has buried enormous amounts of carbon, much of it the residues and shells of living organisms, in calcareous rocks such as chalk and limestone, and in coal seams and oil deposits. This rock-bound carbon can be released by chemical weathering of rocks, which is influenced by soil chemistry and by life. It is also released in volcanic eruptions, and by burning of fossil fuels.

\end{tcolorbox}
\end{figure}

\section*{The soil aggregate}

Where all this comes together is in the \index{soil aggregate}soil aggregate, a tiny clump or clod of mineral particles---sand, silt, clay---held together by plant roots, fungal threads, and the sticky snots, glues, and slimes secreted by plants, bacteria, algae, fungi, and animals. These conglomerations of mineral substances and carbon-rich organic substances form complex sponge-like textures with abundant pores of different sizes and shapes and an enormous, wettable surface area per unit of volume. It's somewhat like a dense city block full of multistory apartments and businesses full of plumbing, wiring, cabinets, clothes, and books, which has acres of surface area, an enormous variety of spaces and habitats within it, a variety of life and behaviors, and a constant traffic of matter and energy.

Because most of what holds the aggregates together is not readily soluble in water, they are able to maintain pore space and structure even when it rains, and thus absorb and hold a good deal of water, releasing it slowly along with dissolved nutrients to plants, trapping loose particles and nutrients, and filtering pollutants. The energy of generations of sunlight, especially that of the water cycle and carbon cycle, is held in suspension by the soil aggregate with its carbon-rich biofilms and cements.

Aggregates form when an abundance and diversity of plants, bacteria, fungi, and animals---which require water, air, energy, and mineral nutrients---produce, and continue to produce, the tendrils and cements that hold the particles together, as well as the armor or skin against sun, wind, and the pounding impact of raindrops. 

\begin{wrapfigure}{L}{6cm}
\includegraphics[width=6cm]{pics/rhizosphere.jpg}
\caption*{Strong, sticky aggregation around grass roots, likely with abundant glomalin-forming mycorrhizae.}
\vspace{-1em}
\end{wrapfigure}

Without these, soil aggregates can come apart fairly quickly. Common bacteria use the exposed aggregate-binding glues and slimes as food, oxidizing the carbon into atmospheric carbon dioxide, leaving loose particles. Raindrops falling through the air are flattened blobs, sometimes with dimples on the underside which act like explosive hollow-point bullets when they hit bare soil, detaching particles from larger clumps, puddling the particles, collapsing whatever pores are left and sealing the soil surface so that water runs off rather than infiltrates. Erosion, sedimentation, flooding, loss of life in and on the soil, and further disintegration of soil aggregates follows. On drying, these puddled surfaces typically harden into a crust, difficult for seedlings to penetrate as well as paving the way for the next runoff event. You can readily observe these conditions in a well-used dirt driveway, in many often-plowed fields, or any situation where soil remains bare and exposed for long periods.

Tillage such as plowing or discing tends to break down soil structure both by physical breakage and by exposure to oxygen, which is usually in short supply in the smaller soil pores. This oxygen promotes the consumption and breakdown of many forms of soil organic carbon by common bacteria. Tillage is also analogous to opening the draft on a wood stove, or turbocharging an engine. The oxidation reactions are accelerated, which yields more power and energy as well as increasing the availability of plant nutrients. Tillage will often increase crop yields, but often at the expense of degrading soil structure (compaction, lack of pore space, tendency toward anaerobic fermentation) and nutrient availability in the future.

Tillage has often been compared to a major disaster for the soil aggregates---to use the city metaphor, it would be equivalent to a severe earthquake followed by both flooding and fire. 

\begin{wrapfigure}{R}{6cm}
\centering
\vspace{-2em}
\includegraphics[width=5cm]{pics/elephant.pdf}
\end{wrapfigure}

Reinforcing feedbacks abound, most of them vicious. Land degradation and soil compaction can occur at every scale, from the square foot to the 40-acre parcel to thousands or millions of acres. With increasing drought severity come more fires. Fire, flooding, or drought can take away soil cover, kill vegetation, and create unfavorable conditions for soil aggregates, leading to increases in heat-trapping gases such as water vapor and carbon dioxide in the atmosphere, which may contribute to more flooding and drought, loss of more soil aggregates, increasing fire and weeds, compaction, and more runoff of fertilizers and contaminants. Because of these vicious circles, we may think of land degradation as inevitable or unstoppable.

\section*{Agriculture's tragic role}

Humans have been destroying soil aggregates since the beginning of agriculture.\label{tragedy} This long-running, civilization-crashing tragedy has come about through our pursuit of sustenance, accompanied by interlocking and unintended side effects. \textit{Conquest of the Land Through 7,000 Years}, a report by Walter C. Lowdermilk from his travels during the 1930s in the Middle East and China, was one of the first to outline the tragic history of agriculture for a popular audience. Lowdermilk often described his work as reading ``the records which farmers, nations, and civilizations have written in the land.'' There was always the possibility that ``some unheralded genius may have already found the solution to our problem, a solution in whole or in part if we know what we are looking for.'' \textit{Conquest} is available online.

Many traditional practices expose soil, and to keep it exposed. For a long time, humans have lit fires. Used repeatedly, burning exposes soil to ready oxidation of its organic matter and the loss of the fungi and other life that help grow it and retain water. Today fire is as popular as ever, and widely used to remove crop residue, dormant and dead grass, and woody debris. On most continents humans continue to use fire to get rid of residues and maintain desired plants and vegetation communities.

Plowing, used repeatedly to prepare seedbeds and reduce competition from weeds, exposes soil, interrupts photosynthesis, and destroys soil aggregates faster than they can form. Where soils were too wet to plow, we've drained them with ditches or drain tile, often followed by repeated plowing, which has exposed wet soils and wetlands to massive oxidation of their organic matter, resulting in the disappearance into atmospheric carbon dioxide of gigatons of soil carbon and meters of carbon-rich peat over millions of acres. The wholesale destruction of forests has had some similar effects.

We've changed the behavior of the grazing animals, both wild and domestic. The animals are both more absent and more stationary---a deadly, soil-exposing combination for grasslands and soils that evolved in seasonally dry environments under large herds that were sometimes massed and moving.

Many farmers and graziers accept or defend soil-degrading agriculture as a matter of survival, at least short term---to satisfy the landlord, to feed the family, the village, the world. Some people simply don't notice, seem not to care, or don't think they have much influence. Today many people show more concern for food safety or animal welfare, and soil health or function has been a difficult issue for environmental movements to recognize or get a grip on.

To grow our food, we've been using our technology to interrupt sunlight energy moving into soils, short-circuiting and accelerating water cycling, carbon cycling, nitrogen cycling, phosphorus runoff, and all that depends on the slow release of water and nutrients from the soil aggregates. 

But potent and transformative as this problem-solving approach has been for increasing crop yields, it cannot grow soil aggregates. We manage against what we don't want, whether it is weeds, pests, erosion, compaction, or low fertility. When our solutions worsen the situation or become new problems, such as the presence of agrochemicals in food and soils, we then manage against these, using labels such as organic or GMO-free. Every new strategy becomes a criticism of the old, which arouses defensive behavior, and the subsequent struggle to advance our positions keeps us from expanding our frame or context or looking to outgrow or transcend the problems. In the U.S. for example, our responses to the awkward fact of soil degradation have institutionalized a zero-sum conflict between production of food and fiber on the one hand and conservation and environmental protection on the other. One side tries to make an ineffective system more efficient, while the other tries to make it more benign, and we lose sight of the possibility of growing an agriculture that was effective at enhancing and maintaining soil structure and function.

Too often the stalemate results in too many plants on welfare, too many animals in prison, too many microbes dead, and too many farmers out of business. This can have serious consequences for us because we are dependent on their work---the most powerful and creative planetary force---for everything we take for granted. With less photosynthesis, there is less biological energy, less carbon, flowing into soils and helping to maintain their structure. We are increasingly subject to floods, drought, impaired water quality, local and global climate change, and in many areas we are in a downward spiral of dependency on increasingly expensive agrochemicals, fossil fuels, irrigation, genetic alterations, and even subsidies to grow poorer and poorer food by fewer farmers, because our compacted soils do not store nearly as much water or carbon, or contain as much biodiversity, as we need them to. Unintended consequences include nutrient runoff into lakes and streams, biodiversity loss, sedimentation, flooding, salt-soaked soils, manure lagoon overflows, dust storms, refugee crises, and even the collapse of communities and civilizations. 

\section*{Exceptions and outliers}

There have been exceptions to agriculture's destructive effects on soil. There are (and have been) river valleys such as the Nile where floodplain agricultural soils, even if badly treated, were periodically replenished and renewed by flood-borne sediment and nutrients from eroding uplands. In many areas, people have built extensive terraces to prevent soil loss and allow soil aggregates to form, and devised crop rotations and pasture leys to minimize the loss of soil structure from tillage. 

From the dawn of agriculture there have very probably been farmers and graziers, perhaps in isolated pockets and margins, who have found ways to enhance and grow soil structure and fertility while continuing to raise food and fiber. These innovations have generally not come from the centers of power, but from the margins, and for the most part they have not been widely adopted.

As agriculture has become less locally oriented, more commercial, and has used more inputs, it has also become the major environmental issue. Some creative and entrepreneurial farmers and ranchers are figuring out how to recombine these problems into opportunities, to farm in nature's image. They have been putting a creative, synergistic understanding of living systems at the center of their search for lower input costs and a better quality of life. They are working \textit{with} more than \textit{against} the power and creativity of the biosphere, using rain and sunlight captured by plants, as well as animals and other organisms, to build soil aggregates, soil health, resiliency, diversity, and effective water cycles. They are tapping into reinforcing feedbacks, where two plus two often yields more than four. They are co-creating results with the greater powers, climbing and combining seven generations of sunlight. But they are a small minority, and often difficult to recognize through the lenses of a zero-sum conflict between efficiency and environmental protection.

\section*{Soil health principles: a whistle blown}

About a dozen years ago a few soil conservation people in North Dakota got tired of trying to conserve what they increasingly saw as a degraded resource. They were also frustrated with the ``best management practices'' method of advising or teaching farmers, which bypassed or superceded the creativity and intelligence of some farmers in the area, some of whom were reorienting themselves toward managing wholes rather than parts, and growing soil structure rapidly with diverse mixtures of cover crops without any tillage. Guided also by a broadly emerging understanding and rediscovery of the role living organisms (from microbes to large grazing animals) play in developing and sustaining soil aggregate structure and function, they developed the following principles of soil health.

\begin{enumerate}
\item\textbf{ minimize tillage}
\item \textbf{keep soil covered}, with plant residue and growing plants
\item use \textbf{diversity}, such as mixes of species and rotations
\item have \textbf{living roots} in the ground as long as possible
\item \textbf{integrate livestock} into the operation
\end{enumerate}

Others have supplemented these principles, adding for example \textbf{know the context of your land}. 

How well does the agriculture around you align with these principles, and what is the evidence---such as bare soil, diversity of crop plants, or the occasional evidence of livestock or grazing---from which you are drawing your conclusions?

Though their roots can be found in agricultural writings from before the Second World War down through Allan Savory's holistic management, these principles remain a radical challenge to our human reflex to activate ourselves \textit{against} what we don't want. Our institutions, agencies, and organizations, our habits and reflexes of citizen and environmental activism and organizing, are oriented around threats, problems, and blame. Though many people within these systems understand this and would like to go beyond those limits, the structures, peer pressures, and incentives (and our fears of their power) often curtail change.

Many previous prescriptions for a more sustainable agriculture were a list of don'ts or prohibited practices, such as the organic certification standards which prohibited various types of chemicals. These principles flew in the face of what the U.S. Department of Agriculture, the input sectors, and the land grant universities have been advising for decades---large, short-season monocultures without soil cover, producing feed for tightly confined cattle, hogs, and poultry. 

Now adopted by the Natural Resources Conservation Service (formerly the Soil Conservation Service) of the U.S. Department of Agriculture, these principles are proving challenging to that agency itself, with its traditions of program delivery and its alignment with research and input sectors who have advocated the opposite for generations. They are challenging our habit of convergent problem-solving, to our reliance on best management practices, to our orientation toward technical fixes. They are challenging for much of agriculture, for much of soil science, for our habits and methods of policy formation. 

\section*{The politics of change}

\epigraph{Does a civilization fall when the soil fails to produce, or does a soil fail only when the people living on it no longer know how to manage their civilization?}{Charles Kellogg, 1938 Yearbook of Agriculture}

\noindent When innovations (particularly transformational ones) come from the margins, it is difficult for the centers of social power and influence to see them as anything but threats. Cultural identity, habit or reflex, positional power, commitment to a position, sunk costs, and ego defenses---often invisible to their owners---may keep people out of position.

When we encounter information and advocacy around an issue, we orient ourselves by taking a position, whether it's to agree, disagree, or dismiss. We may modify our actions. But transformational change---at the level of behaviors and beliefs, of causes rather than symptoms---requires some kind of personal participation. As Wendell Berry wrote in \textit{The Unsettling of America} (1977):

\begin{quotation}\noindent Institutional solutions . . . necessarily fail to solve the problems to which they are addressed because, by definition, they cannot consider the real causes. The only real, practical, hope-giving way to remedy the fragmentation that is the disease of the modern spirit is a small and humble way---a way that a government or agency or organization or institution will never think of, though a person may think of it: one must begin in one's own life the private solutions that can only in turn become public solutions.\end{quotation}

\begin{wrapfigure}{L}{6cm}
\centering
\vspace{-2em}
\includegraphics[width=5cm]{pics/elephant.pdf}
\end{wrapfigure}

Particularly in an industrialized urban society where most people are not farmers or live directly from the land, the opportunities to participate in the soil health opportunity, and thus in most of the major issues of our time, are camouflaged. Soil health or watershed function become slippery and evasive if you try to define them in terms of what you are against, or in terms of what they are not. Though a common enemy or threat is often assumed to have the effect of uniting people, a brief history of any of the issues on the elephant, such as climate or land degradation, suggests instead that managing \textit{against} often leads to fragmentation, where diversity of opinion becomes a threat to effective action.

We live inside the circle of life, a flow of energy from the sun. These soil health principles are about effective energy flow, about the carbon cycle buffering and slowing water cycling, and maintaining effective and non-leaky nitrogen and phosphorus cycling. 

If we focus too far up the flow of sunlight on global metrics such as atmospheric carbon dioxide concentration or global average temperatures, whatever efforts we make are unlikely to have globally measurable effects in our lifetimes. There is little or no accountability for results. If we focus too far down the flow---on our own actions or beliefs, for example---it is easy to deceive ourselves about actual results. For example, I might assume that I'm improving my soil if I perform approved ``carbon farming'' practices this season. But my outcomes may not meet my expectations, and there may be unrecognized factors that contribute to the actual outcomes I get. In both cases we end up flying nearly blind, without real feedback or accountability.

An alternative is to \textbf{monitor in the middle}: gauging and developing a shared intelligence on water cycle function and carbon cycle function, including diversity and behavior of organisms, at a human management scale such as a lot, an acre, a small watershed, and at a human time scale such as 1--15 years. We could gauge the effects of management, so as to learn what works. The geolocated quantitative field methods in the second part are examples. 


\begin{figure}
\includegraphics[width=\textwidth]{pics/change1.pdf}
\caption*{Most agriculture is degrading soil. Cultural identities, enormous sunk costs, and huge market shares heavily influence the research and input sectors. There have long been outliers that maintain or enhance soil structure while growing food, but who do not attract the majority and often disappear.}
\includegraphics[width=\textwidth]{pics/change2.pdf}
\caption*{With pressure for change over environmental issues, there is equal resistance, defending these identities, sunk costs, market shares. \textbf{Positions are stable, frozen, entrenched}. Fantastic sums are spent treating symptoms. Trust is low, blame is rampant. Outliers join in cacophonies of competing advocacies, but there is no accountability around results, and most are out of position, or disoriented, in relation to what we all need. Intelligence is not shared. Nobody learns.}
\includegraphics[width=\textwidth]{pics/change3.pdf}
\caption*{With the growth of a \textbf{shared, broadly participatory intelligence} on soil health and watershed function, positions and advocacies soften, trust, new possibilities, and larger contexts emerge, along with accountability for results based on shared evidence.}
\end{figure}


\chapter{A different science: \\from guarded knowledge to a learning system}

\epigraph{Don't just learn \textit{about} nature. Learn \textit{from} nature.}{Janine Benyus, \textit{Biomimicry}}

\noindent The emerging, or re-emerging realization that our serious issues and challenges revolve around soil structure and function is awkward indeed. As a society we're out of position, disoriented. Our science and decision making have been focused on parts and problems, and driven by narrow incentives. We haven't recognized the importance of soil structure and function, and policy hasn't been effective at enhancing it. (Since 1931 the U.S. Department of Agriculture has spent \$294 billion in 2009 dollars on soil conservation, yet the effectiveness of these expenditures is widely doubted, even within the agency.)

On earth, we live in an enormous flow of sunlight energy. The metabolisms and choices of countless diverse but self-motivated living organisms turn this flow into something highly complex, turbulent, and even sensitive. There are opportunities for small forces to influence large forces such as carbon and water cycling, which come together in soil structure and function. 

We may recognize these opportunities, but commonly we recognize them as problems that we can try to solve. For example, we often hear this question, or something like it:

\textbf{Question 1.} How can we increase soil carbon? What practices, policies, or species should we use? 

This is our default style, aiming at problems, the stripes on the elephant. But in situations of complexity, these questions steer us down the path of blame, power struggles over the ``right'' answers, and low trust. We set up expensive wars on symptoms.

Alternatives are not always obvious or readily accessible. But there is a simple trick, and an old one. Instead of thinking forward from the problem (low or depleted soil carbon), think backward from the result we need, such as:

\textbf{Question 2.} What would need to happen, what conditions would need to exist, for soil carbon to increase, for the ``deep topsoil future'' or the ``soil carbon sponge'' envisioned by Abe Collins or Walter Jehne?

We are often asked Question 1. But our Soil Carbon Coalition was formed around Question 2, which can no longer remain a purely technical, scientific, or policy question, because our recognition of opportunities is heavily filtered by our beliefs, and in particular our beliefs about what's possible and what's impossible. Many people believe that soil cannot be grown or improved quickly, or that possibilities for improvement are narrow and expensive. They may believe that management cannot increase soil carbon. Or, they may believe that certain practices or procedures infallibly do so.

These beliefs often come from experience, and this can become self-reinforcing: if you believe it's not possible, you're right. Doing what you're doing may continue to yield the same results. Our beliefs are typically resistant to efforts by others to educate us or change us, and often resistant to documented, quantitative evidence. Data or information can be easily discounted if it's from a different place, culture, or economic situation: It won't work here. You can't get there from here.

\begin{quotation}\small
\noindent The practice of framing possibility calls on us to use our minds in a manner that is counterintuitive: to think in terms of the contexts that govern us, rather than the evidence we see before our eyes. It trains us to be alert to a new danger that threatens modern life---the danger that unseen definitions, assumptions, and frameworks may be covertly chaining us to the downward spiral and shaping the conditions we want to change. (\textit{The Art of Possibility}, by Rosamund Stone Zander and Benjamin Zander)
\end{quotation}

Our work around Question 2 began with a hypothesis: that a basic precondition for soil carbon increase was that people believed it to be possible. And from there we started the Soil Carbon Challenge, beginning to collect the time-series, open data and local evidence that this belief might grow on.

We quickly got spread thin. Most people, institutions, organizations, and agencies were not oriented around Question 2, and our reflexes and habits of policy formation, technical assistance to agriculture, decision making, even meeting styles kept us in Question 1. Nor was there much capacity to collect or share time-series, open data and evidence of carbon and water cycle change, particularly with the huge swell of innovation going on in farming and ranching. Nor can current research methods capture the complex (positive or negative) impact of every change we make in our dance with photosynthesis, soil microbiology, and the carbon, water and nutrient cycles. Why not, instead, view the whole landscape as a potential learning opportunity, and engage anyone interested as participants in the inquiry and creative process?

This different science is already happening, mostly around environmental problems and species, though often with institutional and professional leadership. And it is beginning to emerge around soil health and watershed function.

When we're dealing with simple problems with single fixes (Schumacher's convergent problem-solving), diversity of opinion and participation can be a liability. But with complex issues with multiple feedbacks at every scale, diversity of opinion and participation can be an asset. This participatory science or learning system may resemble institutional science in use of similar methods or tools, but it can be oriented differently:

\begin{itemize}

\item[] emphasizes possibility more than prediction, using shared evidence as a medium 
\item[] emphasizes repeatability more than defensibility of conclusions
\item[] local more than global
\item[] transparency of data and methods
\item[] the entire landscape (as well as the spectrum of human creativity) can be a learning laboratory, not just small research stations testing standardized practices
\item[] oriented more around people than structured around scientific disciplines, conventions, or apparatus
\item[] best way to predict the future is to create it; creation is now
\item[] large, adaptable contexts, wholes more than parts
\end{itemize}

As a society, we can recognize and learn from important innovations and successes in enhancing watershed function and soil health by combining professional research on changes over time with more participatory data collection on open, citizen-usable, interactive maps.

We can be asking different questions: how can we work \textit{with} nature, with these powerful and creative forces, to achieve a better quality of life? How might we co-create what we need with the circle of life, the carbon cycle? How might we learn what works, and discover or recognize possibility?

Because these are different questions, they drive a different design of observation and experiment.
In addition to theoretical knowledge, classifications, or some kind of ``know-that,'' we can be seeking know-how. Because we are in a situation involving complex feedback loops with all our major challenges, the best way to predict the future is to create it. We are discovering that much of what we thought was relatively fixed and unchanging, is moving and changing, often in response to our influences, and there is resistance, conflict, fragmentation, and confusion around these changing realizations.

\begin{wrapfigure}{R}{6cm}
\centering
\vspace{-2em}
\reflectbox{\includegraphics[width=5.6cm]{pics/calvin.png}}
\caption*{The biosphere's power is vastly greater than that of our technology. However, the system is an open, turbulent flow, not a static equilibrium. Over time, smaller power can influence the force and direction of larger power. Globally, the influence of human technology on biosphere power has tended to work \textit{against} nature, largely inadvertently. With effective systems for feedback and learning, we will have the opportunity to work \textit{with} nature's processes and cycles.}
\end{wrapfigure}

Part 2 of this guide provides examples of how to set one or more benchmarks or fixed plots from which a time series of multiple observations can be made in order to detect and measure change. In effect, a baseline measurement can serve as its own ``control'' in an experiment carried out by biosphere processes, human management decisions, and weather over time. The best places to run these experiments will be where sedimentation, erosion, volcanic depositions, sinkholes, or the human equivalents such as land planing, pipeline construction, or massive fill operations, are less likely to happen.

This can be easier than mapping landscape function or soil carbon over a field or land parcel, but it takes patience. Results are not instant. The longer you wait between observations, the greater your chance of detecting and measuring change, and recognizing possibility.

This guide is about observing and measuring work (force times distance or force against resistance). Work cannot be measured directly, but only by \textbf{comparing the situation before with the situation after.} Recall the example with foot-pounds: In order to quantify foot-pounds, we need to know the weight of the object, the height or level at which the work started, and the current or ending level. In other words, a time series, like recording the heights of a growing child on a door jamb.

A time series requires repeatable measurements. If you are recording the height of a growing child wearing high-heeled shoes against a door jamb, and next time barefoot, the time series or trend line may not mean much. Especially when dealing with soils, soil structure and carbon content, and plant growth, repeatability has four ingredients. Coordinating these ingredients, whether for a research project or a citizen-science effort, can be a considerable challenge.

\begin{enumerate}
\item \textbf{Location.} Soil for example is not well mixed, and has considerable variability over the scale of millimeters as well as meters and miles. If you are sampling soil to diagnose a problem or to sell an amendment, casual sampling may suffice, but in order to confidently detect change over time, the locations of soil samples, biomass samples, infiltration measurements, and soil cover assessments (especially on permanent pasture) must be accurately determined and findable after a span of years. The next chapter will show you how.

\item \textbf{Measurement that is accurate enough, yet practical and affordable.} If an onsite measurement or sampling requires a large, expensive effort, or expensive apparatus and rare expertise, it is less likely to be repeated. 

\item \textbf{Open data.} Many measurements and a great deal of monitoring data are not repeatable because the baseline or prior data is no longer easily accessible or easily interpretable. The person who did the original or previous measurements is gone, and the raw data might not be available. Or, the data may be wrapped in impenetrable jargon, or kept from being shared or computable by its structure or formatting. When data is kept private for any reason, the difficulties of subsequent access are multiplied. But when raw data is openly shared on the internet, for example on interactive maps, repeatability is enhanced because people can see or discover the opportunity of repeating a prior measurement. On the door jamb where you might record the growing height of a child, there's a visible line plus a name and a date.

\item \textbf{People willing and able to collect and record the data.} There are many social, economic, and technical factors that can limit capacity. If the only way to record data publicly is in some rigid data schema that is inaccessible to all but a few, the available capacity for building a dataset is limited.

\end{enumerate}


\begin{figure}
\begin{framed}
\includegraphics[width=\textwidth]{pics/pyramids.pdf}
\small
\begin{minipage}{0.48\textwidth}
\textbf{Problem-oriented:} when dealing with complexity, this structure turns risky. Problems become wedge issues, or shift according to their definitions. Positions and advocacies tend to fragment, with the best becoming the enemy of the good, and diversity of opinion or research results becoming a threat. Greater participation adds topheaviness and urgent need for control, so trust declines. Many people working in these systems want the kind of results suggested by the right-hand pyramid, but they feel trapped (or believe they are trapped) by the urgencies, incentives, and low trust that are inherent in this structure and orientation.
\end{minipage}
\hfill
\begin{minipage}{0.48\textwidth}
\textbf{Opportunity-oriented:} Discoveries and recognitions of success can emerge from time-series data, along with the possibility of effective action to foster soil health and watershed function. This localized, specific, evidence-based sense of success in working with nature's processes and cycles can be solidified and amplified with greater participation in time-series data, both collecting and sharing. ``Nonconforming'' and noisy data can become part of a shared intelligence or learning system without threatening effective, creative local action. Growing participation at the base, such as collaborative monitoring projects, contribute to trust and shared purposes.
\end{minipage}
\vspace{1em}

There is little question that the structure on the left is our default setting, and we often reflexively organize against what we don't want, against what we fear. This usually works well in well-defined issues and in mechanical situations. It comes up short when we're dealing with true complexity, with multiple overlapping feedbacks. We will not be able to abandon it entirely, nor should we.

But we can begin to add the dynamics on the right, in order to address the many complexities we face.

Is policy likely to shift before public opinion does? Where are the opportunities for leadership? How might we gauge the effectiveness of a policy, practice, or program? What could you do, or how could you position yourself and your group, to learn how to manage and work with the complexity of carbon and water cycling?


\end{framed}
\end{figure}
\normalsize

\pagestyle{myheadings}
\renewcommand{\chaptermark}[1]{\markboth{#1}{}}

\makeoddhead{myheadings}{\includegraphics[height=15pt]{pics/elephantoutline1.pdf}}{\footnotesize{Some field methods}}{\thepage}
\clearpage
\Huge{Part 2: Some field methods}
\normalsize
\thispagestyle{empty}

\addcontentsline{toc}{part}{Part 2: Some field methods}

\vspace*{3 em}
\begin{figure}[h]
\centering
\includegraphics[width=.9\textwidth]{pics/transect.pdf}
\end{figure}
\clearpage

\epigraph{What we observe is not nature in itself, but nature exposed to our method of questioning.}{Werner Heisenberg}

\noindent Arthur Eddington, the British astronomer whose observations in 1919 during a solar eclipse gave confirmation to Einstein's general relativity theory, described a common scientific approach as casting a net into the ocean, and making an inventory of the catch. The scientist concludes that no sea-creature is less than two inches long, and that all sea-creatures have gills.

This guide is by no means exhaustive or definitive. There are many ways of tracking change over time on landscape function and biological work, with new understandings, new possibilities and techniques emerging all the time.

We can make observations on landscape function at \textbf{different scales}, ranging from a single point on a landscape, to a field, to a watershed, to an entire continent or planet. Observations at the point or small-area scale can add value and context to observations made at larger scales, and vice versa. The first few chapters in this section describe point or small-area field methods, and later chapters will describe larger-scale observations such as streamflow, remote sensing of photosynthesis and soil cover, and harvest.

Point observations can be highly particular, with all the advantages of detailed attention to observable features and changes. Point observations highlight the variability from place to place, reflecting more localized factors, and sometimes may be the only choice. However, limitations of time and resources usually guarantee a small sample size in relation to the overall land, and it is often difficult to make reliable conclusions about large areas from small sample sizes. 

Observations over larger areas, such as the variability of streamflow with precipitation, are summations of many small-area processes. They may not reflect the variability of these smaller-scale processes. So where possible, it is best to do some small-scale and large-scale observations. They will complement each other.

The field procedures described here have all been tested, but revisions and improvements are ongoing. Check \url{http://soilcarboncoalition.org/guide} for the latest downloadable versions, as well as links to mobile apps for collecting data.\index{updates}

\begin{figure}
\begin{tcolorbox}
\setlength{\parskip}{.7em}
Are the observations you are making repeatable, with accuracy? Try to imagine repeating the observation you are making now, several years into the future, using the notes and data that you are recording. 

What are the sources of inaccuracy or error in your measurements and observations, and how can you address them? How confident are you that what you are recording can be re-observed with accuracy so as to detect change (or the absence of change)?

%WHEN I STARTED THE CHALLENGE I LACKED CONFIDENCE
\end{tcolorbox}
\end{figure}

The following chapters outline some methods of observing and recording data on landscape function and biological work. We start with establishing a reference location called a transect, and then describe some repeatable observations that depend on the transect for location, enabling the growth of a times series so as to learn what's possible and as feedback to management.

\begin{figure}
\begin{tcolorbox}
\setlength{\parskip}{.7em}
\textbf{atlasbiowork.com: \\a flexible coordinate system for locating and sharing repeatable observations}

\url{https://atlasbiowork.com} is a flexible and simple browser-based data-entry app that works on mobile devices such as smartphones and tablets, and laptop or desktop computers. Using it can help you correlate location, date, and other details of your observations so that they might be repeated, and to save this correlation on an open, mapped database. When you upload your data, it becomes visible to others on a map.

It is usable in the field, and can be used without a network connection if you choose ``Add to home screen'' while on a network connection. When you again have network connectivity, you can upload your data to the atlasbiowork.com server, where it is viewable on a map. Development is ongoing, and as of this writing it is a ``web app'' (browser-based), highly usable, and undergoing modifications and improvements.

The various forms available for data entry may be helpful as prompts or checklists as well.
\end{tcolorbox}
\end{figure}

\begin{figure}[h]
\begin{framed}
\small The Field Methods section of this guide was prepared by Didi Pershouse and Peter Donovan of the Soil Carbon Coalition, with the expectation that the reader will make their data publically available on our interactive map database. See \url{http://soilcarboncoalition.org}

Feel free to contact us with your data and any questions about monitoring: 

Peter Donovan: info@soilcarboncoalition.org 541-263-1888

Didi Pershouse: ecologyofcare@gmail.com  802-785-2503

\end{framed}
\end{figure}



\chapter{Site selection and location}
\section*{Purpose}

Accurately measuring change over time in a landscape (amidst variation in soils, management, vegetation, and weather) can be a challenging task. Good location is essential to repeatability of your measurements. An excellent and traditional way to designate and map fixed locations in a changing landscape is with a \textbf{transect}: a straight line across the surface of the earth, for example one determined by two points. This transect can be found again by using a sketch map, measuring tape, and compass, as long as you have chosen landmarks that are unlikely to change. 

By itself, a transect may not be a useful observation, but it establishes a repeatable reference location or site for multiple types of observations over time---in particular, for soil sampling and analysis, and observations of soil cover in areas where perennial plants grow. It is also a good reference location for water infiltration measurements. If your observation plans include these, it is a good idea to start with a transect.

If your observations are going to be limited to earthworm counts, brix (sugar content), residue or soil cover in annual croplands, or if you are measuring change in a small space such as an urban backyard, a full transect may not be needed. For these, approximate location using a street address, description and/or hand drawn map, and latitude and longitude from common GPS receivers---such as those in many smartphones---may be good enough.

\section*{Selecting a location}

The general location you select for a monitoring transect depends on your purpose. In measuring soil carbon change, for example, where the number of locations I am measuring is small (the usual situation) I try to locate a transect according to some or all of the following:

\begin{itemize}
\item[] where a manager is interested in trying to enhance or grow soil carbon
\item[] where slope, aspect, and vegetation is representative of larger areas
\item[] where soil series, or crop yield (some people have maps for both of these), is somewhat typical of larger areas
\end{itemize}

Unless you plan to have many sampling locations, you may wish to avoid areas that may receive atypical management or treatment, such as: edges of fields, where equipment turns around, fences, watering troughs, corrals, driveways, trails and heavily trafficked areas, pipeline corridors, or areas of rapid erosion or deposition of soil. You may also wish to avoid stream channels or gullies, ridgetops, or areas of uncharacteristic slope or aspect.

It is an excellent idea to do a short interview with the people who make management decisions on the land you are monitoring, and choose locations with them. With this interview, you can adapt your purpose, foresee and perhaps avoid practical difficulties, and gain insight into the history and context of the land and its management that you won't already have. While you're at it, note some of this down, and record some contact information for future use. Also check with the manager(s) to see if posting open data on your measurement results is OK with them.

How long is a transect? It depends on your purpose, the observations you intend to make, and the variability you encounter in the land. For a soil sampling plot only, a transect may need only be long enough to reach from a landmark or measuring point into an appropriate area for sampling. If you are establishing some repeatable soil cover observations in a perennial pasture with lots of variation, it may make sense to stretch your tape to its full length, or combine two or three tape lengths along the same line, to sample some of this variability. You can use one transect or line to cross field or pasture boundaries, and locate multiple observation sites along it.


\section*{Transect materials}
\begin{checkboxlist}
 \item 200-foot (60-meter) open-reel fiberglass measuring tape (a shorter one may serve).
\item Some means of pinning both ends of your tape to the ground. Foot-long (60p) nails will work, as will some tent pegs (with small hooks) or surveyor's chaining pins.
\item Sighting compass. Compass apps in smartphones sometimes work well, but often get out of calibration. Be sure to set it to magnetic north. Try checking the accuracy of your smartphone compass with those of others!
\item One handheld mobile device (smartphone) with a camera, and apps already downloaded for a GPS that shows latitude and longitude in decimal coordinates. 

%(Note: GPS apps have widely varying accuracy. Try to find one that gives you accuracy to within 20 feet or less.)
\item Signboard for photos. Two 9 $\times$  12-inch erasable whiteboards, hooked loosely together with wire ties through two holes on one edge, make an adaptable signboard that can be stood on the ground at a variety of angles, or hung over the D-handle of an upright shovel. A clipboard with paper and markers can be used in a pinch.
\item Thick dark markers (erasable if using whiteboard, permanent if using paper) for marking signboard.
\item (optional) Two or more 4-foot $\frac{3}{8}$-inch pointed fiberglass temporary electric fence posts are useful items when laying out a transect. I cut mine to 1 meter in length with a hacksaw, tape a brass cartridge over the end, and wrap the stick at intervals with black tape to make a handy, visible marker and \textbf{alignment stick} that can be pounded into hard ground with a hammer.
\item ``Rite in the Rain'' all-weather notebook for drawing maps and recording data, and/or a notebook or paper on a clipboard
\end{checkboxlist}


\begin{figure}
\begin{tcolorbox}
\setlength{\parskip}{1em}
\subsection{Warm-up exercise for learning groups} 

The best way to practice making sketch maps, and learn tape and compass skills, is to divide into small groups, go to different areas in a park or field (or even indoors) and practice drawing a map locating a specific point on the land, using a tape and compass.

One team then hands their map to another team and watches as they try to find the point. This will make it clear whether they have successfully defined landmarks, and mapped the transect (and the point along it) in a way that someone could find it in the future. For this exercise, it is helpful if there is some way of confirming the destination point in an open landscape, such as burying a small (natural or biodegradable, in case it's not found) object just below the surface.  For example, a rock with an X written on it. 
\end{tcolorbox}
\end{figure}


\section*{Landmarks, reference points, and markers}

Each situation may suggest different ways of mapping and marking your transect. There are three basic strategies: 1) coordinates from GPS receiver and compass bearing, 2) lines of sight and landmarks, and 3) permanent markers. In order to find a transect several years after it has been recorded (perhaps by someone else), we recommend the first, plus the second or third. Using all three is best.

\subsection*{1. GPS coordinates and compass bearing}

Use a GPS (global positioning system) receiver to record locations. Most smartphones and many tablet computers have these, some cameras do, and there are many types of handheld GPS receivers. For most smartphones and tablets, you will need some kind of app to get GPS coordinates. There are many of them, including our own \url{https://atlasbiowork.com} which will record location. Be sure to use latitude and longitude in decimal degrees, which is the most popular and least ambiguous format for coordinates, and will give you a pair of numbers such as 41.39857, -82.23948 as your location, or the location of a landmark where you are standing.


\begin{figure}
\begin{tcolorbox}
\setlength{\parskip}{.7em}
\textbf{GPS.} Most consumer-grade GPS receivers do not have accuracy better than about 20 feet or so. An interesting and revealing exercise for a group is for everyone to record coordinates at an easily identifiable point location such as a corner of a large building or a road intersection. Then enter these coordinates in an online mapping utility such as Google Earth, and see how far they differ. By yourself you can mark a spot in open land, record the location, go away several hundred yards, and try to find it again using only the GPS receiver.

It's a good idea to check your GPS record (and your hand-drawn map of a transect) against an online mapping utility such as Google Earth or Maps. Enter your recorded coordinates in the search box and see how it corresponds to your transect. 
\end{tcolorbox}
\end{figure}


\subsection*{2. Lines of sight and landmarks}

\textbf{Two points determine a line.} An excellent way to fix and record the location of a transect line is to find a line of sight, a visual alignment, that connects two near-permanent landmarks or reference points. For example, can you find a line where a utility pole lines up with something more distant, such as a mountain peak, notch, steeple, or cell phone tower?

Good landmarks will be visible whether or not trees have leaves on them, when nearby trees have grown taller, or when a field may have 8-foot corn. Do not use trees (which grow, die, and fall over) or temporary buildings unless there are no alternatives. In some areas and seasons, low clouds and fog may often obscure distant features. These can make good landmarks:

\begin{itemize}
\item corners or peaks of permanent buildings, grain bins, or silos

\item corners, centers, or edges of culverts, bridges, roads

\item peaks, gaps, or saddles in ridgetops

\item large boulders or rock formations with definable corners or points

\item well-head covers, irrigation risers, water spigots

\item utility poles

\item cell phone towers, radio masts, permanent flagpoles

\item treated fenceposts such as railroad ties, or steel posts in concrete
\end{itemize}

Large objects with defined edges from certain directions, such as the vertical side of a grain bin, rock formation, or road cut, or where one ridge disappears behind another, may be excellent landmarks and reference points but only when they are combined with another. The right-hand side of a silo, for example, depends on your angle of view.

A permanent rectangular building can sometimes supply two landmarks or reference points---two corners---if you line up with a side. Sometimes this alignment is easily visible, sometimes not. Check your alignment's sensitivity by stepping to one side and another. 

\subsection*{3. Permanent markers}

If there are no landmarks or lines of sight available, or if they are too far away to measure from with a tape, you should install your own in the form of permanent markers. Markers can also add good confidence to relocating the transect.

What markers you choose, and where you put them, will depend on the situation. You want something that:

\begin{itemize}
\item will remain in place, and can be found again
\item is practical, durable, and easy to install
\item does not pose a hazard to humans, animals or wildlife, or vehicle tires
\item is compatible with management plans and tools, such as tillage, forage harvest, seeding, or even burning or wildfire
\end{itemize}

Check your choice of markers with the manager(s) of the land. 

Markers are best placed at the ends of the transect, rather than at measurement locations or other midpoints, which can be relocated using the measuring tape. This is because markers, like landmarks, may influence soil changes locally. Livestock or game may use an upright post as a rub, concentrating hoof impact. Also, record the location of these markers with accuracy on your map, as they will be more difficult for future monitoring teams to find than visible landmarks.

In large pastures that aren't going to be drilled or tilled, heavy steel disk blades are very good, secured to the ground through the center hole with $\frac{3}{8}$-inch steel rebar, bent over so as not to be a hazard. A white plastic bucket lid, though it may not last more than a few years in full sun, is also a visible marker. It can be fastened to the ground with ring-shanked pole-barn nails, but won't last where wild pigs are common. In big open country, a bucket lid or disk blade can also be marked with a $\frac{3}{8}$-inch fiberglass post through the center, which might last years depending. 

Lacking a disk, we recommend a 20--24-inch section of $\frac{3}{8}$-inch steel rebar with one end bent in a J or eye. This can be driven flush to the ground with a couple of feet of aluminum wire affixed at one end for easier relocation, or through a piece of an aluminum can. A heavy plastic stake driven almost flush to the ground is good where fire or mechanical disturbance won't occur. You may want to create signage that explains the reason for the marker---for example a piece of plastic saying ``monitoring site landmark: do not remove'' attached to the wire. 

In cultivated fields a piece of steel such as rebar can be driven into the soil below the tillage layer, but this will then require a metal detector to relocate it. Steel that is at the surface can often be found, if your search area is small and vegetation is not high, by moving a magnetic compass over the area and watching for deflection.

Where a stake may be removed by agricultural machinery, it may also work to put a marker, or to improvise one, on a fenceline or tree row, and to extend the transect to it. Large stones or rock cairns are sometimes effective where machinery or animals won't scatter them.

\section*{Laying out your transect}

\begin{enumerate}
\item Review your purpose. How does the general location you have selected meet that purpose?

\item What landmarks or reference points are available? Look for lines of sight that align near points with far ones, or a point in front of you with one behind.

\item Choose your zero point, a spot to start measuring from. If one landmark is within 150 feet of the plot (or within easy walking distance so that you can accurately stretch the tape out several times), then designate that landmark as your ``0'' (zero) point to start measuring from. Write it down, e.g.: ``Point 0 = north side of well-head.'' (If neither landmark is near you, you will select a zero point from the choices in the ``Lining up a transect'' diagram.)

\item Secure the zero end of the tape to the ground with a stake or pin.

\item Walk out towards the second landmark, or compass direction, unrolling the tape, past the spot(s) where you may want to take measurements of soil, soil cover, or infiltration.
\item Stretch the tape between the two ends so that it is flat, and being sure that it is aligned exactly with your chosen landmarks. (Having two or three people working together on this to stand back and tell you when it is aligned is useful, but with alignment sticks you can also do this solo.)
\item Secure the far end of the tape into the ground by driving a tent peg through the handle, being sure it is stretched straight. In windy conditions, keeping the tape on the ground or touching plants can help keep your tape straight.
\item Decide where along the tape your photo point will be. Take a compass reading along the transect line, \textbf{away from the zero point}, and write it down.

\item Begin to draw your map, using a straightish line for the transect, with the zero point clearly indicated. It is conventional to orient your transect map so that north is at the top. If your compass bearing is 180 degrees, for example, your zero point will be at the top, and the transect line will run straight down the page. Add the photo point and its distance from zero, such as ``30 meters.''
\begin{figure}[h]
\centering
\begin{minipage}{.75\textwidth}
\includegraphics[width=\textwidth]{pics/transectline.jpg}
\caption*{Example line photo along a transect from a photo point. The axis of the center pivot sprinkler (center of photo) lines up with the tip of Mt. Shasta. If this volcano doesn't blow its top, this transect can be relocated within inches without GPS.}
\end{minipage}
\end{figure}

\item While standing at the photo point, take a GPS reading, using decimal degrees to five places, (e.g. 45.62020, -117.77186). 

\item Choose a name for the transect. For example, the first transect at Hartford School could be named something like HART1.
\item Write a sign, using a thick marker that will show up in a photo. (An erasable white board is useful here so that you can add and change other labels for other photos and measurements). The sign should include today's date, and the transect name or identifier. 

\item While straddling the tape at the photo point, take a photograph along the transect tape from the photo point toward the zero point---with someone holding the sign so that the tape, the sign and your line of sight is visible. (If you are alone you can prop the sign against a shovel or hang it from the shovel handle.) Shoot from eye level. Get a little sky in the frame, but not too much. Shoot straight along the tape, which should divide your photo into two equal halves.

\item Take another photograph along the tape in the opposite direction, also showing your signboard 
\item If your landmarks are distant and hard to see on a photo with a wide-angle view, zoom in and take another. You may also wish to take a photo along auxiliary lines of sight (with three- and four-point alignments, as in the ``Lining up a transect'' diagram.)


\item \textbf{Add some more detail to your map} showing:
\begin{enumerate}
\item Which way North is on the map (most expect that North will be toward the top)
\item The landmarks or permanent markers that you are sighting between. Be sure to describe what the marker is made of (such as metal rod, wooden railroad tie, large stone, cement well-head cover)
\item The zero point
\item Name and rough drawings of one or two other notable features nearby (Connecticut River, Route 5, blueberry shed, telephone pole etc.)
\item The transect line, with compass bearing
\item The photo point, and how many feet or meters it is from point zero
\item The GPS location of the photo point
\item Today's date
\item The name and physical address of the general location (i.e. the name of the school, farm, etc.)
\item A description of the area around the plot (``horse pasture'' ``baseball field,'' ``school garden'' ``front lawn'')
\item The name you have chosen for the plot (e.g. ``MASC1''  or ``HART3'')
\end{enumerate}

\begin{figure}
\includegraphics[width=\textwidth]{pics/sketchmap.pdf}
\caption*{A sample sketchmap of a transect, with a soil sampling grid and soil cover hoops indicated.}\end{figure}

\item {Record anything you know about the history of the land, how it has been managed in the past (with approximate dates) and how it is currently being managed. For example: 

\begin{quotation}\noindent \small 1985--2012: maintained as school lawn, using commercial herbicides and fertilizers.  Mowed once a week. 2012 turned into school garden and started planting diverse vegetable crops until present. 2012--2014: rototilled once a year, adding composted manure from Vaughn Farms each spring. No-till from May 2014 to present. Compost tea made according to Soil Food Web standards added May 2015. Currently no-till diverse vegetable garden using organic methods. See attached document for details of planting and management.\end{quotation}
}
\item Leave your stretched tape in place, and don't trip over it. It is now the reference location for subsequent observations.

\item Before you roll up your tape and leave the site, a) install any permanent markers you have decided on, and b) complete any auxiliary measurements and compass bearings to other landmarks or reference points, and add these to your map. Use the checklist below.

\end{enumerate}
\section*{Checklist: recording your data}

\begin{checkboxlist}
\item Contact person with phone, physical address, and email
\item Details of land management history and current practices
\item Today's date
\item Name or identifier of transect
\item GPS coordinates
\item Compass reading along transect
\item Line photos of transect
\item Hand-drawn map of alignments, distances, bearings, locations, and landmarks
\item Complete any measurements such as soil sampling, soil cover, infiltration, and record locations on your sketchmap
\item Install permanent markers if needed
\item Finish and record distances and bearings to nearby landmarks

\end{checkboxlist}

\chapter{Using the observation hoop to track surface changes}

\section*{Purpose}

We use photographs and written notes about what we see within an ``observation hoop'' to track changes over time---in plant diversity, amount of bare ground, and other observations that can be seen without disturbing the soil---on a single spot on the land. This is an essential step in monitoring the health and function of grasslands and grassy areas such as lawns, hayfields, pastures, etc. (and can be useful even in regularly tilled ground). 

\section*{When}
If you are using a transect, this is typically the second step after setting up a transect. The first hoop observations are typically done at the side of the spot on the tape that will become the center of your 4 $\times$ 4-meter grid for soil sampling.

One or more sets of hoop observations can be done elsewhere along the transect tape---typically at spots that show something different than the first hoop, but which are still representative of the overall ground cover (such as above average, average, below average).

At the Soil Carbon Coalition, we've learned some of our methods from rangeland monitoring techniques such as Land EKG \index{Land EKG} or the ``Bullseye!'' method.\footnote{These methods were developed primarily for rangelands in the Intermountain West, but have some adaptability. For Land EKG, see \url{http://landekg.com}. For ``Bullseye!'', see \url{ http://quiviracoalition.org/images/pdfs/1634-Monitoring_the_Resilient_Ranch_Presentation.pdf }}


\section*{Materials}
\begin{checkboxlist}
\item 94'' circumference hoop (a 94'' section of plastic covered $\frac{3}{16}$-inch wire works best. If the ends are not secured, you can simply shape it into a circle).
\item Mobile device or smartphone with a digital camera and a GPS app already downloaded.
\item Clipboard
\item Pencil
\item Notebook (``Rite-in-the-Rain'' paper is great) and mobile data app: \url{https://atlasbiowork.com}
\item Plastic bag to cover your phone and/or paper.
\end{checkboxlist}

If you are using the hoop as part of a transect (which we recommend) please start with that section of this guide---being sure to draw a map, and record GPS readings and your location along the tape for the photo point of the transect. 
\section*{Photographing the hoop and the sign}
\begin{figure}

\includegraphics[width=.48\textwidth]{pics/hoop.jpg}
\hfill
\includegraphics[width=.48\textwidth]{pics/stepback.jpg}
\end{figure}

\begin{itemize}
\item Place the observation hoop on the ground (on the southern side of the tape if possible) so that one edge touches the tape. Make sure that the ends of the hoop (or wire) are touching to form a complete circle, and that the circle is round, not oval.
\item Write a sign saying the date, the transect name (``CEDA-1'') and the location in feet or meters away from the zero point on the tape.  (Make sure you identify your units!) If it is the plot center, write ``plot center,'' as well as the location in feet or meters. 
\item  Place the sign on one side of it, and take two photographs of the hoop. a) One photograph, looking straight down at the hoop, that includes the sign. b) Step back a few paces and take a photograph of the hoop at an angle that includes the hoop, the sign, and the horizon. Get a little sky in the frame, but not too much.
\end{itemize}
\section*{Recording observations}
\begin{enumerate}
\item {Using your hand, calculate the percentage of bare soil that you can see within the hoop. (The following percentages correspond to an average adult hand. You can adjust accordingly if small children are doing the measuring.) \begin{enumerate}
\item Whole hand with fingers at maximum spread= 6\%
\item Whole hand with fingers not spread = 4\%
\item Bottom of fist = 1\%
\item Thumbprint = $\frac{1}{3}$\% \end{enumerate}
Add up the percentages of bare soil, and record the result. In perennial pastures, record percentages of other definable covers, using subtraction if convenient, such as litter, moss or algae, rock, or plant base.}

\item How many centimeters of dead and decaying plant material (``litter'') do you see on the soil surface?
\item Does the soil surface appear to be sealed off from rain, or can see you open pores?
\item How many different grass species? If you know names of species, write them down too. Otherwise just count the numbers of different types.
\item How many different broad-leafed plants (``forbs'')?
\item How many shrubs or shrub-like plants?
\item How many different tree saplings?
\item Are there mounded clumps of grass that seem to be part alive and part dead?
\item Any of the following: moss, mold, lichen, mushrooms? 
\item Do you see insects, spiders, worms, or signs of their activity?
\item What else do you notice about the soil surface?
\end{enumerate}

\section*{Using a hoop without a transect}
If you are just doing hoop observations, you will want to be sure to record a location in a way that you or someone else can reliably find this spot again, and record the following.

\begin{checkboxlist}

\item Name of observer
\item Today's date
\item Name of plot
\item Contact person with phone, physical address, and email for plot
\item GPS location of hoop, (with at least 5 numbers after the decimal, e.g. 45.62020, -117.77186 and accuracy ``to 16 meters'')
\item Details of land management history and current practices.
\end{checkboxlist}

\begin{figure}
\begin{tcolorbox}
\setlength{\parskip}{.7em}

\textbf{Extra activities}

Move some of the plant litter aside and, using a 5x magnifying loupe, look at the soil surface. If you see any spider or insect activity, what work do they seem to be doing?

Using a 5--10x magnifying loupe, look carefully at a plant, flower, lichen, moss, or insect. Draw what you see. 
What else does it look like?

Why do you think it is shaped like that?

Using a field guide, see if you can identify the plants and insects you found in your observation hoop. 

Do some research, and answer one or more of these questions:
\begin{itemize}
\item Do any of the plants you found have medicinal (or other special) uses? 
\item Are they edible? If so, what do they taste like?
\item Are they wild plants or do you think someone planted them?
\item What work do those plants do in this landscape?
\item What does the presence of those plants tell you about the ecosystem you are observing? Where is it at in succession? What does it tell you about what is happening with the water cycle?
\item What work do the insects you observed do in this landscape? What do they eat?
\item Why do you think these things are living in this location? What might it tell you about the health of the soil and surrounding ecosystem?
\end{itemize}
If you have time, you can create a drawing of some or all of what you see. This might help you notice details.
\end{tcolorbox}
\end{figure}

\chapter{Infiltration: How water moves into soil}

\section*{Why} 

There are many ways and devices for measuring infiltration---the vertical movement of water into the soil's pores. What follows is one method, using simple equipment, that fosters \textbf{an increasing awareness of water cycle function for participants.} If you do this carefully and attentively, it can also furnish \textbf{a repeatable observation that might show change in soil structure and water cycle function.} Infiltration is not simulation of rainfall, which is varied, but when done consistently gives a good picture of \textbf{how soil accepts water,} reflecting the structure and stability of soil aggregates, and how well well larger pores stay open when water soaks the surface.

Infiltration often shows lots of variability: varying across short distances, with soil surface conditions, with soil moisture, and by season of the year and stages of plant growth, which influence pore and aggregate structure and activities of soil organisms. Measurements are probably best done when plants are well established and growing. If you make infiltration measurements at other times of year, it may be best to repeat them at approximately the same season---and this holds true for most indicators of landscape function and soil health, which can vary seasonally.

A good measurement may take an hour or two, but you will not be occupied the whole time, so it is easily combined with other observations at or near the same location.

For a short video on water infiltration, see \url{http://soilcarboncoalition.org/infiltration}


\section*{What you will need}

\begin{checkboxlist}
\item {\textbf{infiltration rings:} 5 or more sections of metal pipe, mild steel is best, about 6 inches (15 cm) in diameter and about 6 inches long. These should be about $\frac{3}{32}$ inch (1.5 mm) in thickness, or about U.S. 12-gauge which is slightly more than a tenth of an inch in thickness. One edge should be beveled sharp at about 45 degrees with a grinder, and can be kept sharp with a file. You need a single reference point on the circumference of each ring for measuring drop in water level. A welded seam will serve, otherwise file a slight notch in the top edge.

For routinely soft soils, aluminum irrigation pipe works well, but steel is better for all-around use. In very soft soils a plastic 5-gallon bucket with the bottom cut off can substitute for several steel rings.
}

\item \textbf{15 cm, 6-inch steel rule} with millimeter scale for measuring ring diameter, and depth to water surface

\item \textbf{hammer or mallet} for driving rings into soil. A 4-pound deadblow hammer is best as these 6-inch rings can be hard to drive. A hand sledge also works well.

\item \textbf{wood blocks} to place between hammer and rings for driving into soil, about $2 \times 6 \times 12$ inches is good. These may split or break so it's good to have replacements. Or, make a driver out of steel or very tough plastic.

\item \textbf{water container} such as a 5-gallon collapsible jug for transporting water to the site. For five 6-inch rings, plan on having at least 3 gallons available.

\item \textbf{mobile device} for recording data, getting location, and taking photos. You may use the infiltration form at \url{https://atlasbiowork.com} on a mobile device, which records timings as well as other data. If you are using a mobile device, it is also good to have backups: \textbf{auxiliary battery for mobile device, stopwatch, notebook, pencil.}

\item \textbf{plastic bag or plastic wrap,} about 12 inches square, for protecting the soil surface from the pour of water from your measure

\item \textbf{flat file} for sharpening beveled edge of infiltration rings

\item \textbf{signboard} such as a pair of 9-inch by 12-inch whiteboards tied together at one edge, for labeling photos, along with dry-erase marker and a rag 

\item {\textbf{some kind of measure} for 1 inch or 2.54 cm of water for your infiltration ring. A ring with an inside diameter of 6 inches will need about 15 ounces or 441 cubic centimeters (milliliters) of water. To calculate 1 inch or 2.54 cm of water for your ring, measure the inside diameter. Half the diameter is the radius $r$. The formula for the area of a circle is area = $\pi r^{2}$. The volume of water needed is the area multiplied by 1 inch or 2.54 cm.

\textbf{Metric example:} a round section of pipe has an inside diameter of 14.8 centimeters. The radius is half that, 7.4 centimeters. Using the formula $\pi r^{2}$ we get $3.14159 \times 7.4 \times 7.4 \times 2.54 = 437$ cubic centimeters or milliliters. (1 fluid ounce = 29.6 milliliters.) 

If you are measuring in inches, you will get cubic inches. For example, a pipe with a 6-inch inside diameter has a radius of 3 inches.  $\pi r^{2} = 28.3$ square inches, and the same in cubic inches when multiplied by our depth of 1 inch. Multiply cubic inches by .554  to get fluid ounces, thus our measure in this case should be 15.7 fluid ounces. 

Using a measuring cup or graduated cylinder, pour that amount of water into a clear plastic bottle, set it on a flat surface, and mark the water level with an indelible marker, or cut the plastic bottle at the water surface.  This will give you a measure of one inch of water for your ring. You may also use a can or other handy container with the correct volume. 
}

\item A serrated knife is always a good thing to have when working with soils that may be sodbound with abundant roots. In very heavy sod, it may be difficult to drive even a well-sharpened infiltration ring into the soil, which may just bounce with the blows. Cutting a slot in the sod, using the infiltration ring as a pattern and guide, may be needed. 

\end{checkboxlist}

\section*{How to do it}
\begin{enumerate}
\item \textbf{Select a location.} This could be a fairly uniform area in terms of soil cover, vegetation, or management, or it might be a transition between these. It is best to place rings within a meter or two of each other to make it easier to watch them, and to record an observation for a small area. 

\item \textbf{Place your rings on the ground, sharp edge down.} If rings are in an approximate line, such as along a transect tape, it is easier to keep track of the order and the timings. You may want to place rings on the variety of soil surfaces in the area that you have selected---such as bare soil, well-vegetated soil, average, and so on.
\item{\textbf{Drive infiltration rings straight into soil} with your hammer and wooden block. In tougher soils and sods your rings must be sharp and you must swing that hammer in order to drive them in straight. Try to get them about halfway in (about 3 inches for a ring that is 6 inches in diameter and 6 inches long).

You may firm the soil with your fingertips around the inside of the ring to take care of slight cracks from driving in the ring, but be careful not to disturb the rest of the soil surface, or the litter or mulch on it. In heavy vegetation it may be necessary to clip the vegetation in the ring area before driving in your rings.}

\item \textbf{Place a plastic bag over the ring, push down the center enough to hold your inch of water, and pour in one measure (one inch) of water.} Slowly tug the plastic out from under the water and start your stopwatch. The purpose of the plastic is to protect the soil surface from scouring by your pour, stirring up sediment which can then settle in and plug the soil pores. 
 
\item \textbf{Time the disappearance of the first inch of water.} Record the amount of time (in minutes and seconds) it takes for the inch of water to infiltrate the soil. Stop timing when the surface is just glistening. If the soil surface is uneven or sloping inside the ring, stop timing when \textbf{half of the surface} is exposed and just glistening. When litter or heavy vegetation is present, use a pencil or similar object to push it aside so as to see whether there is standing water on the soil surface, or it is just glistening. 

\item \textbf{Run an inch of water in the rest of your rings.} Unless the infiltration is very fast (under a minute or two), begin applying an inch of water to your other rings in order, using the \url{https://atlasbiowork.com} infiltration app to keep track of the timings. If using a stopwatch, start each ring at half- or one-minute intervals so you can use one stopwatch for all. For example ring 1 begins at 0:00, ring 2 begins at 1:00, ring 3 at 2:00, and so on. Just be sure you subtract the starting time from the finishing time for each ring in order to tally the elapsed time.

\item \textbf{Add subsequent inches of water.} After the first inch has finished using the half rule, add a second inch, recording the timing using the half rule, and a third and fourth inch similarly. The subsequent inches will generally take more time as more soil pores fill with water, and as pore space collapses.  After several applications of an inch of water, most soils will tend to approach a fairly steady value, reflecting its ability to keep its pores open and take water in a saturated condition.

\item \textbf{Infiltration that doesn't slow down with subsequent inches.}  If repeated applications of an inch of water do not result in slower timings, you may wish to ``saturate'' the soil in the ring with a generous dose of water, let it infiltrate, and try timing another inch or two.

\item \textbf{Recording the drop with the slow ones.}  If the first or second inch is taking over 30 minutes, record the drop in water level in 30 minutes as follows: 1) Add another inch if needed, and measure and record the vertical distance in millimeters between the lip of the ring and the water surface, at the seam or notch in your ring. 2) Set a timer for 30 minutes. 3) After 30 minutes, remeasure this distance, and record the difference in millimeters.


\end{enumerate}

\begin{figure}
\includegraphics[width=\textwidth]{pics/infiltration.jpg}
\caption*{Infiltration rings along a transect.}
\end{figure}



\section*{Recording data}

\begin{enumerate}
\item \textbf{Use the app.} Go to \url{https://atlasbiowork.com} which will enable you to enter data with a smartphone or computer, although we recommend having backup options in case your mobile device's battery dies. Data entered via the app will be public, open, and mapped.

\item \textbf{Record the location} using GPS, and the approximate locations on a located transect if you are combining infiltration with other observations.

\item \textbf{Take a photo} that shows soil surface conditions and vegetation, using a signboard with site identifier and date.

\item \textbf{Describe the site:} its management, its vegetation and soil cover, its slope and aspect.

\item \textbf{Briefly describe the site of each ring} in terms of soil cover and vegetation, especially if they differ.

\item \textbf{Record timings} of each inch of water for each ring. If using paper, this is best done with a grid or table format, with a row for each ring, and a description of its soil surface, plus start/stop or elapsed timings for each inch of water in the columns across. 

\item \textbf{Reflect and learn.} Do you see connections or correlations between the condition of the soil surface (crusting, litter cover, vegetation, arthropod activity, etc.), and the speed of infiltration? What might be the implications for runoff and surface water quality, groundwater or aquifer recharge, biodiversity both above and below ground, or production of food and fiber? What about flooding and drought, or perceived drought? If this is a remeasurement, what is the trend, and how might that reflect management of this land?

\end{enumerate}

\section*{Potential issues}
If the \textbf{water goes down very quickly} (less than a minute) in one of your rings, and not nearly so fast in the others, it is likely that you have created a channel in the soil by rocking the ring back and forth as you pounded it into the ground. Or there may be a cavity such as a soil crack or animal hole under the ring. Note down the time, but also note that it seemed suspiciously fast. If you are only doing one or two rings, you might also want to redo that ring in another location.

In dry conditions, with shrink-swell or cracking clays, water may move in very fast through the cracks. If and when the cracks seal because the clay swells with water, the surface may become nearly impervious to water.

In very hard ground, such as dry, compacted hard clay, seriously rocky soils, or heavy sod, it may be difficult to drive rings halfway into the soil. In sod, cutting the sod with a serrated knife, using the infiltration ring as a guide, will help. Some soils may be difficult or impossible to test with these tools and methods. Do the best you can and note the fact that you haven't driven the rings very far in. Or come back when the soil is wetter and softer.


\chapter{Sampling for soil carbon}

\section*{Purpose}

To extract relatively undisturbed samples of soil from distinct layers of the earth that can be tested for carbon (and other components, if desired). This should be done in a geo-located plot, so that results from that same area can be compared for gains or losses in carbon over time.

\textbf{Related activities:} Sampling for soil carbon should be done after setting up and mapping a transect, and photographing the observation hoop. Typically we also test water infiltration and bulk density at the same time.

\textbf{Brief overview:} After setting up a transect, lay out meter sticks to delineate a 4 $\times$ 4-meter grid. Lay out tarp nearby to create a workspace. Label your bags and soil containers. Take samples from predetermined spots on the grid, slicing each core into three or more samples (representing top, middle, and lower layers) and adding each sample into appropriate container, to mix with other samples from that layer. When you have finished collecting cores from 8 total sampling points then place contents of containers into a clearly labelled bag for transport.  Dry samples and send to laboratory for analysis.

\section*{Materials}
\begin{checkboxlist}
\item Manual soil sampling probe. For ease of use and sturdy quality we recommend the Hoffer 36'' soil sampler. (The open slot should be at least as long as the depth to which you will be sampling.) If you want to sample more deeply, other probes are available.

\item Shovel. In case soil is so compact or rocky that you need to do a pit sampling technique. 

\item Soil probe with a slide hammer. (Optional but very useful in hard, compacted soils.) We recommend the 1'' x 36'' AMS brand soil probe with a replaceable tip (part number 425.52) along with a 12-lb. slide hammer (part number 400.99). Be sure to order at least one extra tip, as rocky soils will damage the tip easily (part number 425.74).  

\item Meter sticks to help lay out the grid (at least 2).  These can be cheap wooden pre-made ones available in hardware stores, or you can make them yourself out of $\frac{3}{8}$-inch fiberglass electric fence posts or other rigid materials. They don't need to have markers on them, they just need to be one meter in length. 

\item A light-colored piece of cloth/tarp that you can put on the ground as a workspace (approximately 2 square meters is plenty). This can also double as a wrap for your long tools.

\item Three sturdy plastic containers that won't tip over, to hold and mix the soil cores while you are extracting them. The kind that are sold for food leftovers---about 8 $\times $ 4 inches and 3 inches high---work fine.

\item Quart or Gallon-sized re-sealable plastic bags, for transporting the soil samples when you are done. The heavier, freezer-style bags with a zipper closure are best. You will need 3 for each plot, but extras are always a good idea.

\item Permanent marking pen for writing on bags

\item Serrated knife, for slicing soil cores and extracting them from the sampler.

\item A rigid flat metal metric measuring stick to lay out next to soil sampler when slicing core samples.  (You could use meter sticks for this, if they are marked with centimeters.)

\item Round metal file for sharpening inside of probe tips while in the field. 

\item A few ounces distilled vinegar: (optional) you can use this to test for inorganic carbon in soils.

\end{checkboxlist}

\section*{Laying out the sampling grid}

\begin{figure}
\centering
\includegraphics[width = \textwidth]{pics/EZgrid.pdf}
\caption*{\textbf{A grid plot layout} provides 24 sampling locations, a meter apart, in a 4 $\times$ 4 meter square around the center point. \label{EZgrid}At each sampling, up to 8 cores can be taken in previously unsampled points. If soil pits are needed, choose one of the outside locations. Use grid locations for bulk density sampling as well. After three samplings, the grid spacing can be expanded from 1 to 1.5 m to provide additional, previously unsampled points.}
\end{figure}


The sampling grid is an imaginary 4 $\times$ 4-meter grid, with 25 potential sampling points located at the corner points where the lines intersect (NOT in the center of each square meter; see illustration). The grid layout enables us to take multiple samples in a compact area, on repeated occasions over time, without resampling a previously disturbed hole.

If you have trouble visualizing this on the ground, you can lay out this whole grid with wire flags and meter sticks around your chosen center point of the transect. Groups may find this helpful the first time that they are doing a plot.

Once the grid is clear in your mind, in the future you can just refer to a picture of it to see the sampling points, and use several meter sticks that you move around as needed to be sure your measurements are accurate.

\section*{Deciding on sampling points}
In an unplowed pasture, field, or yard, we recommend sampling 8 points on the grid, at three or more layers. In a tilled field, 4--6 sampling points on the grid should be adequate. In a forest, you will want to enlarge your grid substantially, in order to get soil samples that represent the larger scale of the variations in forest ecology (i.e. if you take samples under a single pine tree, it may be very different than what you would find under the various trees in the larger area.) 

If you can afford it, you may want to have soil from individual sampling points analyzed, rather than mixing the samples together, especially the first year. The resulting data will establish the variability among samples in your plot, and can be used to gauge the sufficiency of your sampling design. (This may also be useful in the future if you choose to publish your data in a scientific journal.)  In general, however, in testing for variability, we have found that 8 sampling points, combined into three total samples (one mixed sample for each layer) provides a sufficient number to overcome most variability concerns. That is the sampling design we have generally used for our own plots.

For your baseline plot (the first year that you sample), we recommend sampling at points marked with circles (see illustration). The next time you come back to sample, use points marked with squares, etc. 

\section*{Decide on sampling depths}

If you are resampling to measure change over time, you must use the same sampling depths and increments as were used in the baseline or prior sampling. For new sites, sampling depths depend on: 
\begin{itemize}
\item[] \textbf{your purpose}: how relevant or important are deeper layers, and what is the capacity of management to influence them? (Even annual plant roots can reach down to 8 feet or more.)
\item[] \textbf{your equipment}: how easy, practical, or accurate is it to sample to depth?
\item[] \textbf{soil conditions} such as depth to impenetrable rock or subsoil.

\end{itemize}
Because we favor hand tools such as the Hoffer soil sampler for ease of use and sturdiness, our sampling depth has tended to go no deeper than 40 cm (and sometimes only to 30 cm.) We often divide our 40-cm core into three (yes, uneven) layers: 0--10, 10--25, and 25--40 cm. 

\begin{figure}
\begin{tcolorbox}
\setlength{\parskip}{.7em}
\textbf{Types of soil probes.} The hammer probe listed above is bulky to transport, but can go to deeper depths and still provide an intact core with relatively little disturbance. Soil augers go even deeper, but they disturb the soil structure of the sample itself as they extract the core.  You can use a pit method to go as deeply as you like and extract undisturbed samples by digging with a shovel, but that involves quite a bit more labor and creates more soil disturbance overall---not ideal for a plot to which one intends to return, but still the best choice for very rocky soils.  

Optimally, you should sample up to the depth at which you think soil carbon is most likely to be changing, with at least 3 designated layers.  These layers are measured not eyeballed. They will not necessarily correspond with visible soil horizons or color changes, because horizons will shift over time as soil health improves or deteriorates. Likewise, it is best if the division of sampling layers stays the same in future years, to track progress in that location (e.g. 10--25 cm). 

If, over time, you are finding substantial gains in soil carbon at the 25-40 cm depth on a piece of land, it would be worthwhile to start measuring below that.  Likewise, if substantial numbers of perennial plant roots are extending below 40 cm, you may want to consider doing deeper samples in at least a few of the sampling points.

If you are going to sample deeper than 40 cm. and you want to compare your data layers with ours, or if you decide to sample more deeply in future years, you can use the divisions we are using, and add an additional (4th) sampling layer at the bottom, rather than shifting your division of layers into three larger ones. 
\end{tcolorbox}
\end{figure}

\section*{Labeling}
Label the 3 containers, with the layers and the depths to which you will be sampling, such as: A layer: 0-10 cm; B layer: 10-25 cm; C layer: 25-40 cm. (These containers can be re-used, so don't put the plot name or date on them.)

Label each plastic bag with today's date, plot name (4-letters plus a number, e.g. CEDA-1 for the 1st plot at Cedar Circle Farm), and depth of the layer, e.g. 0--10 cm.

\section*{Extracting soil cores with a hand probe}

\begin{figure}

\end{figure}

\begin{enumerate}
\item Using the grid map, lay out the meter sticks to find your first sampling point. 

\item Gently push aside litter (dried or decaying plants) from soil surface to create a spot for hand probe. 

\item Push the hand probe straight down into the ground at your first sampling point.  Try to go a few centimeters deeper than your deepest intended sampling depth. (You may need to give it a few extra shoves to get it down far enough. Do not screw it into the ground or wiggle it back and forth, as that may disturb the sample and/or bend the probe.) 

\item When you have reached the intended depth, give the probe a half turn, then pull upwards, tilting the probe slightly away from the open side as you come up, so as not to lose soil, especially if it is dry and crumbly.

\item Lay the probe on the workspace you created with your piece of cloth, with the handle on your left. 

\item Lay the flat metal measuring stick along the probe. The 0 cm end of the measuring stick now represents the surface of the ground so it should be lined up with the top of the soil sample (on the left).

\item Using the serrated knife, slice the core sample into the desired layers. For example if you slice it at 10 cm, 25 cm., and 40 cm, this will give you three segments: 0--10, 10--25, and 25--40. Make sure the top of the soil (towards the handle) is your starting point for measuring.

\item While cupping your hand or placing the meter stick over the two other layers in the probe (so they don't spill out), tilt the probe so that the sample from the A layer goes into the plastic container labeled ``A layer.''  Use the knife or your finger to help if necessary.

\item Repeat step 6 for the B and C layers, making sure they go into the appropriate containers.

\item Repeat steps 1--7 for each sampling point, adding all the A layer samples into the same A container, etc. (unless you are planning to test each sampling point separately, in which case they should be labelled accordingly and kept separate).

\item Pour the entire contents of the A layer container into the plastic bag, being sure that it is labelled correctly with plot name and layer (e.g. ``CEDA-1A,'' for the combined A layer samples in the plot called CEDA-1). Make sure you also write down the top and bottom measurements of that layer, e.g. ``0--10 cm.''

\item At home, pour out the contents of each bag onto a plastic plate to dry. Attach the empty labeled bag to the plate with tape, clothespin, or use a stapler on the top edge of the bag, and place in an area where nothing will fall onto it. (The sample is considered ``air dried'' when a rebagged sample doesn't fog up when placed direct sunlight---usually about a week of drying, but it varies with temperature and humidity.) 

\item Once the soil is air dried, you can send the labeled samples off to a laboratory. Check with local universities to find out what sort of sampling equipment they have. At the Soil Carbon Coalition we use laboratories that can do elemental analysis for total carbon using dry combustion. Ward Labs in Kearney, Nebraska offers quick service and reasonable prices. \url{http://wardlab.com}

\end{enumerate}

In rocky or gravelly soils, it may be most practical to gather two samples each from four small \textbf{soil pits}\index{soil pit}, taking care to note the size and location of the pits relative to the plot center, so that future sampling can use different locations, and to restore the pits upon completion as fully and carefully as possible. Dig a hole in the ground with at least one vertical side which can be made with a shovel. It should be at least 8 inches wide most of the way to the bottom. With a ruler, measure your depth increments from the surface and insert some kind of metal marker, such as a nail or small knife, into the side of the pit at your division points. Then you can get your samples from the sides of the pit using a spoon or ice cream scoop, again taking care to make each sample representative of the entire layer sampled. If there are rocks or gravel present, do your best to collect a representative sample of fine earth from the top to the bottom of the layer. Don't worry about collecting rocks, roots, or gravel as they will be sieved out during sample preparation.

If you wish to perform other soil analyses on your samples, you should have plenty of soil.  

\chapter{Soil density}

Density\index{soil density} is the oven-dry weight per unit volume of undisturbed soil. Density also reflects compaction, and decreases in density mean more aeration and pore spaces.

Measuring it requires taking a sample of known volume, drying it, and weighing it. The utility of this measurement depends on your purpose. If you are trying to measure tonnage of carbon in a given depth of soil, or change in that tonnage, then density is a needed factor. Density also reflects porosity and compaction.

Because of their lack of accuracy or consistency relative to carbon analysis using dry combustion, density measurements may merely introduce noise to a simple program of monitoring change in soil carbon due to management. Density also reflects porosity or compaction, and changes over time in density can reflect improvement or deterioration in soil structure.

\section*{Equipment}

\begin{checkboxlist}

\item A simple and practical bulk density \textbf{core sampler} can be made out of a section of sturdy steel pipe about 3 inches or so in diameter. Exhaust pipe works well. Cut a section about 4 or 5 inches long, making sure the cuts are true and square. With a file or grinder, bevel the edge on one end from the outside of the pipe toward the inside at about a 45-degree angle. The inside edge should be square and reasonably sharp.

\item Hammer with wood block to drive sampler into soil
\item 15cm steel rule graduated in millimeters, with a pocket clip that can be used as a slide, enabling you to use this as a depth gauge
\item flat pointed trowel, sharpened
\item sample bags and marker

\end{checkboxlist}

\section*{Taking the sample}
\begin{enumerate}
\item  use a trowel or putty knife to prepare a flat plane surface of undisturbed soil near the midpoint of the layer you want to sample, at one of the grid locations. This can be a horizontal or vertical surface, but as you sample greater depths, we prefer horizontal surfaces. 

\item With the block of wood and a hammer, tap the corer square into the flat surface of soil, at least 2 or 3 inches. If the soil surface inside the ring moves inward as you tap, you are deforming the soil and may need to use the clod method described below.

\item The depth of the ring determines the soil volume contained. With a short metric steel rule, take four measurements, evenly spaced around the ring, of the distance between the outer or blunt edge of the ring to the soil surface within. The best steel rule to use is one with a movable slide or shirt-pocket clip, as you are often working in the bottom of a dark pit and can't read it accurately. The clip can be moved with your thumb, allowing you to probe the depth from the rim of the sampler to the soil surface somewhat like a depth gauge, and then remove the rule to read the distance in millimeters.

\item {The average of these four measurements in centimeters, subtracted from the length of your corer, gives you the length of your bulk sample. Multiply this by the cross-sectional area of your corer ($\pi r^{2}$, where $r$ is the inside radius of your corer) to get your volume. Using centimeters, your result will be in cubic centimeters, which simplifies the bulk density calculation. For example, my corer, made from a section of 3-inch steel pipe, has a cross-sectional area of 41.51 cm and a length of 11.0 cm. After I tap it into a flat surface of soil, it protrudes 3.65 cm (average of four measurements around the circle). The length of my sample is 7.35 cm $\times$ 41.51 = 305.0985 which I round to 305.1 cubic centimeters.

It is best to take these measurements \textbf{before} excavating your corer. In some situations you may want to seal the top of your corer with your sample bag and a rubber band so that no soil or other material leaves or enters the corer during excavation.}

\item Write the volume in cubic centimeters, as well as the plot, grid position, and layer identifier, on your sample bag with a permanent marker.

\item With the trowel or sharpened putty knife, carefully excavate the buried sharp end of the corer, so as not to interfere with the soil within it, until you can cut off the sample, flat and flush along the sharp edge of the corer (a serrated knife works well for this). Now you have a known cylindrical volume of undisturbed, uncompacted soil. Push it out of the corer and into your labeled sample bag, taking care to collect the entire sample. This may take a bit of practice.

\end{enumerate}

\begin{figure}
\centering
\includegraphics[width = 4in]{pics/bulkdensity.jpg}
\caption*{A bulk density sampler tapped into a flat surface beginning at 28 cm below the soil surface. The next steps are to measure the inset of the soil surface inside the sampler and calculate the volume, then label the sample bag and excavate the sampler. This photo shows my preferred method of first digging narrow but deep pit, and then doing density sampling adjacent to the pit as a series of steps, where the corer can then be excavated from beneath.}
\end{figure}


\section*{The clod method}

If you cannot take a sample using the this method because of gravel and rocks, or because the soil fractures or crumbles easily when the corer is tapped in, you may need to use the clod method.\index{clod method}\footnote{This clod method is taken from the USDA Soil quality test kit guide, prepared by John Doran.}

At one of the grid sampling locations, prepare a level plane surface of undisturbed soil at the needed depth, about midway down in the layer you are sampling. With the trowel, dig a bowl-shaped hole about 3 inches deep and 5 inches in diameter. Avoid compacting the soil around the hole while digging. Place all of the soil and gravel removed from the hole in a plastic bag.

Put the soil in the plastic bag through a 2-mm sieve and into a clean bucket. Put the sieved soil back into the plastic bag, and keep the gravel and rocks in the sieve. (If the soil is too wet to sieve, you'll need to save it for later, when you can air dry it, sieve it, and account for the volume of gravel by displacement in a graduated beaker or cylinder.)

Carefully line the hole with plastic wrap, leaving excess around the edge of the hole. Place the sieved rocks and gravel carefully in the center of the hole atop the plastic wrap, making sure they do not protrude above the level of the soil surface.

Using the 140-cc syringe to keep track of the volume, fill the hole with water up to the level of the soil surface. The volume of water required is the volume of the sample you have in the plastic bag. Write this volume in cubic centimeters on your sample bag.

\section*{Drying and weighing}

Most soil labs will dry and weigh samples to calculate bulk density. You may also do this yourself, after the field sampling, if you have a gram scale accurate to .1 gram.

After the sample has been thoroughly air-dried, spread it on a microwaveable paper plate of known weight. (Large samples may require more than one paper plate.) Weigh the sample. Dry the sample thoroughly using a microwave at full power for 1--3 minutes depending on the size of the sample. If you smell smoke, you are overdoing it, combusting organic matter! Weigh the sample again, and record the weight. Microwave it again for 15 to 30 seconds. When it no longer loses weight after a short drying cycle in the microwave, it is dry. Record the weight of the dry sample in grams, less the weight of the paper plate of course. The bulk density\index{bulk density} $D$ is

\begin{equation}D = \frac{W}{V}\end{equation}

\noindent where $W$ is the weight in grams and $V$ is the volume in cubic centimeters (even including sieved-out rocks; see below).

You may also oven-dry your density samples in an oven at 250--300$^{\circ}$ F for several hours. If samples are moist to begin with, it will take longer to get them oven dry.

It is important that the bulk density sample be as similar as possible to the carbon samples. If there are rock fragments larger than 2 mm in your bulk density sample, sieve out the rocks over 2mm in diameter and note the volume of the rocks using displacement with a graduated cylinder or beaker. However, and this is important, \textbf{do not subtract the volume of the sieved rocks over 2mm in diameter from your sample volume, but do not include them when weighing your oven-dried sample.} In effect, this assumes that there is no carbon in these rocks, which may or may not be true, but unless you want to grind and analyze the rocks, you are better off just using the lower bulk density figure that results from not weighing the rocks in calculating the tons per hectare of carbon.

%PART 3
\clearpage

\makeoddhead{myheadings}{\includegraphics[height=15pt]{pics/elephantoutline1.pdf}}{\footnotesize{Activities and investigations}}{\thepage}

\Huge{Part 3: Activities and investigations}
\normalsize
\thispagestyle{empty}

\addcontentsline{toc}{part}{Part 3: Activities and investigations}

\vspace*{3 em}
\begin{figure}[h]
\centering
\includegraphics[width=.7\textwidth]{pics/fisherman.jpg}
\end{figure}
\clearpage

\chapter{Water cycle}

\section*{Rainy day walk}

One good way to see how your local water cycle is functioning is to take an investigative walk 1--6 hours after the beginning of a heavy rain. This is a good group activity. Good raingear and rubber boots, an umbrella if it's not too windy, a signboard that will work in wet conditions, a camera, a rain gauge, and several jars with lids to collect runoff water are recommended.

Have your rain gauge ready for the rain storm. Record the rainfall amount when you start the walk and when you end it, and after the storm is over.

Check the places where you expect heavy runoff, and also the places where you don't, where infiltration might be better. Collect runoff water in a few places, such as the downstream end of a culvert, or a ditch or flowing rivulet. How much sediment do you see? You can let the sediment settle for a few days on a shelf and measure and describe the layers of sediment.

What visible features accelerate runoff? Which ones allow more infiltration?

You may use the Photo observation form in the \url{https://atlasbiowork.com} app for taking a few geolocated pictures of the land and the runoff coming off.

What did you learn about your local water cycle and how do you feel about it?

\begin{figure}
\begin{tcolorbox}
\setlength{\parskip}{.7em}
You can use all your senses in making observations about soil health or ecosystem function. Walking barefoot for example will give you lots of information related to the soil health principles: sponginess, temperature, moisture, and soil cover to name a few. 

Many people have developed their sense of smell to gauge approximately whether a freshly broken chunk of moist soil has a predominantly bacterial whiff, or is more fungal. 

Listen for sounds of insects and birds. Some people have even developed highly sensitive sonogram techniques for assessing biodiversity such as \url{http://wildlifeacoustics.com}.
\end{tcolorbox}
\end{figure}

\section*{Rainfall simulators}

Simulating rainfall events on soils can be a great path toward learning the importance of soil biology and structure to water cycle function.

Here's a very simple type of rainfall simulator that can be used to see some effects of soil cover. You can view a video of this demo from \url{http://managingwholes.com/eco-water-cycle.htm}

\begin{figure}
\centering
\includegraphics[width=.85\textwidth]{pics/jugs.pdf}
\caption*{You will need four 2-liter plastic soda bottles or equivalent, four tumblers, the bent hollow handles from two plastic milk jugs or equivalent, a sharp knife, and tar or glue.\\

Cut the tops off two of the bottles, and make small holes in the bottom to allow water to continue downwards. Use tar or glue to caulk the hollow jug handles (or sections of vinyl tubing) as spouts into holes in the bottles about an inch below the top rim. Fill both containers with soil slightly above the level of the bottom of the spouts, exactly the same on both jugs. It is important to get this right, and some tamping or firming is usually required so water does not immediately wash away the soil surface leading to the spout. Put a layer of dead plant material (litter) on the soil in one of the containers, and this can also be firmed somewhat. Use the bottoms of the remaining bottles under the soil containers. Place tumblers under the runoff spouts, and simulate a rainstorm by pouring 8 ounces (1 cup) of water into each container, preferably through a container with small holes in the bottom to simulate rain.\\

Note: With soils of limited hydraulic conductivity such as clays, which are very slow to let water infiltrate, this experiment may not show much. It works much better with soils that absorb water readily, because in effect this demonstration accelerates or exaggerates the differences. Because the surface litter slows the runoff, it has more chance to infiltrate into the underlying soil. Where infiltration rates are slow, the litter will not create much of a difference, in part because the rainstorm is so rapid.\\

But if you are using soils that can accept water readily, just do it where people can see it over the next few minutes. The results require little comment. Second and third rainstorms can also be instructive. Over time, the rates of evaporation from covered and bare soils can be observed.\\

Calculating the volume of runoff or groundwater recharge per inch of rainfall per unit of surface area is straightforward and shocking. The apparatus can also be used for basic comparisons of water-cycle function using cylindrical cores of topsoil, with existing plants, litter, and humus, taken out of the ground with a posthole digger or bulb planter.}

\end{figure}

\begin{figure}
\centering
\includegraphics[width=.9\textwidth]{pics/watercycling1.pdf}
\end{figure}
\begin{figure}
\centering
\includegraphics[width=.9\textwidth]{pics/watercycling2.pdf}
\end{figure}
\begin{figure}
\centering
\includegraphics[width=.9\textwidth]{pics/watercycling3.pdf}
\end{figure}
\begin{figure}
\centering
\includegraphics[width=.9\textwidth]{pics/watercycling4.pdf}
\end{figure}

\end{document}
